\documentclass[]{article}

\usepackage{psfrag}
\usepackage{epsfig}
\usepackage{amsmath}
\usepackage{color}
\usepackage{rotating}
\usepackage{pifont}
\usepackage{soul}
\usepackage{xspace}
\usepackage{subfigure}
\usepackage[margin=2cm]{geometry}

\newcommand{\Checkmark}
    {\ding{51}\xspace}
\newcommand{\XSolid}
    {\ding{53}\xspace}


\begin{document}

The function families used for the Lyness-Kaganove\footnote{J.~N.~Lyness and J.~J.~Kaganove, {\em A technique for comparing automatic quadrature
routines}. Comp.~J.~20, 2, 170--177, 1977.} test are
%
\begin{eqnarray}
    \int_0^1 |x-\lambda|^\alpha\,\mbox{d}x,\quad & & \lambda \in [0,1],\ \alpha \in [-0.5,0] \label{eqn:val_sing} \\
    \int_0^1 (x>\lambda) e^{\alpha x}\,\mbox{d}x,\quad & & \lambda \in [0,1],\ \alpha \in [0,1] \label{eqn:val_disc} \\
    \int_0^1 \exp (-\alpha|x-\lambda|)\,\mbox{d}x,\quad & & \lambda \in [0,1],\ \alpha \in [0,4] \label{eqn:val_c0} \\
    \int_1^2 10^\alpha / ((x-\lambda)^2 + 10^\alpha)\,\mbox{d}x,\quad & & \lambda \in [1,2], \alpha \in [-6,-3] \label{eqn:val_peak} \\
    \int_1^2 \sum_{i=1}^4 10^\alpha / ((x-\lambda_i)^2 + 10^\alpha)\,\mbox{d}x,\quad & & \lambda_i \in [1,2], \alpha \in [-5,-3] \label{eqn:val_4peak} \\
    \int_0^1 2\beta(x-\lambda)\cos(\beta(x-\lambda)^2)\,\mbox{d}x, \quad & & \lambda \in [0,1],\ \alpha \in [1.8,2], \label{eqn:val_oscill} \\
    & & \beta = 10^\alpha/\max\{\lambda^2,(1-\lambda)^2\} \nonumber
\end{eqnarray}
%
where the boolean expressions are evaluated to 0 or 1.
The integrals are computed to relative precisions of $\tau=10^{-3}$,
$10^{-6}$, $10^{-9}$ and $10^{-12}$ for $1\,000$ realizations
of the random parameters $\lambda$ and $\alpha$.
The results of these tests are shown in Table~\ref{tab:LKtest}.
For each function, the number of correct and incorrect integrations
is given with, in brackets, the number of cases each where a warning
(either explicit or whenever an error estimate larger than the requested
tolerance is returned) was issued.

The functions used for the ``battery'' test are
%
\begin{equation*}\begin{array}{rclrcl}
    f_1 & = & \textstyle \int_0^1 e^{x} \,\mbox{d}x &
    
        f_{14} & = & \textstyle \int_0^{10} \sqrt{50}e^{-50\pi x^2} \,\mbox{d}x \\
        
    f_2 & = & \textstyle \int_0^1 (x>0.3) \,\mbox{d}x &
    
        f_{15} & = & \textstyle \int_0^{10} 25 e^{-25x} \,\mbox{d}x \\
        
    f_3 & = & \textstyle \int_0^1 x^{1/2} \,\mbox{d}x &
    
        f_{16} & = & \textstyle \int_0^{10} 50(\pi(2500x^2+1))^{-1} \,\mbox{d}x \\
        
    f_4 & = & \textstyle \int_{-1}^1 (\frac{23}{25}\cosh(x) - \cos(x)) \,\mbox{d}x \quad &
    
        f_{17} & = & \textstyle \int_{0}^1 50(\sin(50\pi x)/(50\pi x))^2 \,\mbox{d}x \\
        
    f_5 & = & \textstyle \int_{-1}^1 (x^4 + x^2 + 0.9)^{-1} \,\mbox{d}x &
    
        f_{18} & = & \textstyle \int_{0}^\pi \cos(\cos(x) + 3 \sin(x) + 2 \cos(2 x) + 3 \cos(3 x)) \,\mbox{d}x \\
        
    f_6 & = & \textstyle \int_0^1 x^{3/2} \,\mbox{d}x &
    
        f_{19} & = & \textstyle \int_0^1 \log(x) \,\mbox{d}x \\
        
    f_7 & = & \textstyle \int_0^1 x^{-1/2} \,\mbox{d}x &
    
        f_{20} & = & \textstyle \int_{-1}^1 (1.005 + x^2)^{-1} \,\mbox{d}x \\
        
    f_8 & = & \textstyle \int_0^1 (1+x^4)^{-1} \,\mbox{d}x &
    
        f_{21} & = & \textstyle \int_0^1 \sum_{i=1}^3 \left[ \cosh(20^i(x-2i/10))\right]^{-1} \,\mbox{d}x \\
        
    f_9 & = & \textstyle \int_0^1 2(2 + \sin(10\pi x))^{-1} \,\mbox{d}x &
    
        f_{22} & = & \textstyle \int_0^1 4\pi^2x \sin(20\pi x) \cos(2\pi x) \,\mbox{d}x \\
        
    f_{10} & = & \textstyle \int_0^1 (1+x)^{-1} \,\mbox{d}x &
    
        f_{23} & = & \textstyle \int_0^1 (1+(230x-30)^2)^{-1} \,\mbox{d}x \\
        
    f_{11} & = & \textstyle \int_0^1 (1+e^x)^{-1} \,\mbox{d}x & 
    
        f_{24} & = & \textstyle \int_0^3 \lfloor e^x \rfloor \,\mbox{d}x \\
        
    f_{12} & = & \textstyle \int_0^1 x(e^x-1)^{-1} \,\mbox{d}x &
    
        f_{25} & = & \textstyle \int_0^5 (x+1)(x<1) + (3-x)(1 \leq x \leq3) \\
        
    f_{13} & = & \textstyle \int_{0}^1 \sin(100\pi x) / (\pi x) \,\mbox{d}x &
        & &  \quad + 2(x>3) \,\mbox{d}x \\
        
\end{array}\end{equation*}
%
where the boolean expressions
in $f_2$ and $f_{25}$ evaluate to 0 or 1.
The functions are taken from Gander \& Gautschi (2008)\footnote{W.~Gander and W.~Gautschi, {\em Adaptive quadrature --- revisited}, Tech.~Rep.~306, Department of Computer
Science, ETH Zurich, Switzerland, 1998.} with the following
modifications:
%
\begin{itemize}
    \item No special treatment is given to the case $x=0$ in
        $f_{12}$, allowing the integrand to return $\mathsf{NaN}$.
    \item $f_{13}$ and $f_{17}$ are integrated from 0 to 1 as
        opposed to 0.1 to 1 and 0.01 to 1 respectively, allowing
        the integrand to return $\mathsf{NaN}$ for $x=0$.
    \item No special treatment of $x<10^{-15}$ in $f_{19}$ allowing
        the integrand to return $-\mathsf{Inf}$.
    \item $f_{24}$ was suggested by J.\ Waldvogel as a simple yet tricky test
        function with multiple discontinuities.
    \item $f_{25}$ was introduced in Gander \& Gautschi, yet
        not used in the battery test therein.
\end{itemize}


\begin{sidewaystable}
    \begin{center}\begin{scriptsize}
    \begin{tabular}{l|c|c|c|c|c|c|c|c|c|c|c|c|c|c|c}
        $\tau=10^{-3}$ & \multicolumn{3}{c|}{\tt quad} & \multicolumn{3}{c|}{\tt quadl} & \multicolumn{3}{c|}{\tt quadgk} & \multicolumn{3}{c|}{\tt integral} & \multicolumn{3}{c}{\tt quadcc} \\
        $f(x)$ & \Checkmark & \XSolid & $n_\mathsf{eval}$ & \Checkmark & \XSolid & $n_\mathsf{eval}$ & \Checkmark & \XSolid & $n_\mathsf{eval}$ & \Checkmark & \XSolid & $n_\mathsf{eval}$ & \Checkmark & \XSolid & $n_\mathsf{eval}$ \\ \hline
            \input{testsuite_1e-03.tex}
    \end{tabular}\end{scriptsize}\end{center}

    \begin{center}\begin{scriptsize}\begin{tabular}{l|c|c|c|c|c|c|c|c|c|c|c|c|c|c|c}
        $\tau=10^{-6}$ & \multicolumn{3}{c|}{\tt quad} & \multicolumn{3}{c|}{\tt quadl} & \multicolumn{3}{c|}{\tt quadgk} & \multicolumn{3}{c|}{\tt integral} & \multicolumn{3}{c}{\tt quadcc} \\
        $f(x)$ & \Checkmark & \XSolid & $n_\mathsf{eval}$ & \Checkmark & \XSolid & $n_\mathsf{eval}$ & \Checkmark & \XSolid & $n_\mathsf{eval}$ & \Checkmark & \XSolid & $n_\mathsf{eval}$ & \Checkmark & \XSolid & $n_\mathsf{eval}$ \\ \hline
            \input{testsuite_1e-06.tex}
    \end{tabular}\end{scriptsize}\end{center}

    \begin{center}\begin{scriptsize}\begin{tabular}{l|c|c|c|c|c|c|c|c|c|c|c|c|c|c|c}
        $\tau=10^{-9}$ & \multicolumn{3}{c|}{\tt quad} & \multicolumn{3}{c|}{\tt quadl} & \multicolumn{3}{c|}{\tt quadgk} & \multicolumn{3}{c|}{\tt integral} & \multicolumn{3}{c}{\tt quadcc} \\
        $f(x)$ & \Checkmark & \XSolid & $n_\mathsf{eval}$ & \Checkmark & \XSolid & $n_\mathsf{eval}$ & \Checkmark & \XSolid & $n_\mathsf{eval}$ & \Checkmark & \XSolid & $n_\mathsf{eval}$ & \Checkmark & \XSolid & $n_\mathsf{eval}$ \\ \hline
            \input{testsuite_1e-09.tex}
    \end{tabular}\end{scriptsize}\end{center}

    \begin{center}\begin{scriptsize}\begin{tabular}{l|c|c|c|c|c|c|c|c|c|c|c|c|c|c|c}
        $\tau=10^{-12}$ & \multicolumn{3}{c|}{\tt quad} & \multicolumn{3}{c|}{\tt quadl} & \multicolumn{3}{c|}{\tt quadgk} & \multicolumn{3}{c|}{\tt integral} & \multicolumn{3}{c}{\tt quadcc} \\
        $f(x)$ & \Checkmark & \XSolid & $n_\mathsf{eval}$ & \Checkmark & \XSolid & $n_\mathsf{eval}$ & \Checkmark & \XSolid & $n_\mathsf{eval}$ & \Checkmark & \XSolid & $n_\mathsf{eval}$ & \Checkmark & \XSolid & $n_\mathsf{eval}$ \\ \hline
            \input{testsuite_1e-12.tex}
    \end{tabular}\end{scriptsize}\end{center}
    \caption{Results of the Lyness-Kaganove tests for $\tau=10^{-3}, 10^{-6}, 10^{-9}$
        and $10^{-12}$. The columns marked with \Checkmark and \XSolid indicate the number
        of correct and incorrect results respectively, out of 1\,000 runs. The numbers
        in brackets indicate the number of runs in which a warning was issued. The column
        $n_\mathsf{eval}$ contains the average number of function evaluations required
        for each run.}
    \label{tab:LKtest}
\end{sidewaystable}

\begin{sidewaystable}
    \begin{center}\begin{scriptsize}
    \begin{tabular}{l|c|c|c|c|c||c|c|c|c|c||c|c|c|c|c||c|c|c|c|c}
        & \multicolumn{5}{c||}{$\tau = 10^{-3}$}& \multicolumn{5}{c||}{$\tau = 10^{-6}$}& \multicolumn{5}{c||}{$\tau = 10^{-9}$}& \multicolumn{5}{c}{$\tau = 10^{-12}$} \\
        $f(x)$ & {\tt quad} & {\tt quadl} & {\tt quadgk} & {\tt integral} & {\tt quadcc}
        & {\tt quad} & {\tt quadl} & {\tt quadgk} & {\tt integral} & {\tt quadcc}
        & {\tt quad} & {\tt quadl} & {\tt quadgk} & {\tt integral} & {\tt quadcc}
        & {\tt quad} & {\tt quadl} & {\tt quadgk} & {\tt integral} & {\tt quadcc} \\ \hline
        \input{battery.tex}
    \hline
    \end{tabular}
    \end{scriptsize}\end{center}
    \caption{Results of battery test for $\tau=10^{-3}, 10^{-6}, 10^{-9}$ and $10^{-12}$.
        The columns contain the number of function evaulations required by each integrator
        for each tolerance.
        For each test and tolerance, the best result (least function evaluations) is in bold
        and unsuccessful runs are stricken through.}
    \label{tab:battery}
\end{sidewaystable}

\begin{figure}
    \begin{center}
        \epsfig{file=scatter_1e-03.pdf,width=0.48\textwidth}
        \epsfig{file=scatter_1e-06.pdf,width=0.48\textwidth}
        \epsfig{file=scatter_1e-09.pdf,width=0.48\textwidth}
        \epsfig{file=scatter_1e-12.pdf,width=0.48\textwidth}
    \end{center}
    \caption{Scatter-plots of the results of the Lyness-Kaganove
        testsuite for each tolerance. Each point represents one of
        the test functions (Equations~\ref{eqn:val_sing} to \ref{eqn:val_oscill}).
        Its location is determined by the relative number of function evaluations
        (on the $x$-axis) and the relative number of correct evaluations
        (on the $y$-axis). Algorithms in the top left corner are (relatively)
        efficient and reliable whereas algorithms in the lower right corner are
        (relatively) inefficient and unreliable.}
    \label{fig:results}
\end{figure}

\end{document}
