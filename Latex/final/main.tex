\documentclass[12pt, a4paper, twoside]{book}

%% import packages
\usepackage[english ]{babel}
\usepackage[utf8x]{inputenc}
\usepackage{amsmath,amsthm}
\usepackage{amsfonts}
\usepackage{graphicx}
\usepackage[export]{adjustbox}
\usepackage{float}
\usepackage{lipsum}
\usepackage{bbm}
\usepackage{bm}
\usepackage{empheq}
\usepackage{mathtools}
\usepackage{algorithm,algpseudocode}
\usepackage{booktabs,longtable}
\usepackage{array}
\usepackage{siunitx}
\sisetup{output-exponent-marker=\ensuremath{\mathrm{e}}}


\usepackage{hyperref}
%\usepackage{appendix}

%\DeclareMathOperator*{\sup}{sup}
\DeclareMathOperator*{\argmax}{arg\,max} % argmax operator

% remark theoremstyle
\newtheoremstyle{break}
{\topsep}{\topsep}%
{\itshape}{}%
{\bfseries}{}%
{\newline}{}%
\theoremstyle{break}
\newtheorem{remark}{Remark}[section]

% problem theoremstyle
\newtheoremstyle{problemstyle}  % <name>
{10pt}  % <space above>
{10pt}  % <space below>
{\normalfont} % <body font>
{}  % <indent amount}
{\bfseries\itshape} % <theorem head font>
{\normalfont\bfseries:} % <punctuation after theorem head>
{.5em} % <space after theorem head>
{} % <theorem head spec (can be left empty, meaning `normal')>
\theoremstyle{problemstyle}
\newtheorem{problem}{Problem}[section] % Comment out [section] toremove section number dependence
\newtheorem{definition}{Definition}[section]
\newtheorem{theorem}{Theorem}[section]
\newtheorem{proposition}{Proposition}[section]
%\newtheorem{remark}{Remark}[section]

% create command for variance in math mode
\newcommand{\Var}[1]{\operatorname{Var}\left[#1\right]}
% first look for .png (low-resolution) than for .pdf (high resolution)
\DeclareGraphicsExtensions{.jpeg,.png,.pdf}


%\includeonly{Chapters/AssetClass_modeling} 
\begin{document}


%----------------------------------------------------------------------------------------
% COVER PAGE
%----------------------------------------------------------------------------------------
\pagestyle{empty}
%\input{Cover/cover}

%----------------------------------------------------------------------------------------
% ABSTRACT
%---------------------------------------------------------------------------------------

%% Set page numbers of the introduction to roman  
\frontmatter
%\pagestyle{fancy}
%
\begin{abstract}
	Before the 1950s, managing other people's money was a discipline as far away from being scientifically dictated as it could ever get. For instance, before Markowitz's revolutionary paper \textit{Portfolio Selection}, diversification was not a universally recognized practice in Asset Allocation. Markowitz's work paved the way for new ideas from different scientific and academical fields to influence Finance and Asset Allocation in particular.
	
	Continuing along this line, in this thesis cutting-edge results in Stochastic Reachability (which is a concept belonging to the theory of Control Systems) are employed to tackle the Asset Allocation problem. In particular, once an investor has specified his risk profile and a target return, the model will output an optimal investment strategy having the feature of maximizing the probability of reaching the target return while keeping the risk under control. This strategy will exhibit a \textit{contrarian} behavior, namely it prescribes to buy risky assets (in order to achieve a riskier position) when performance is down and to sell them when performance is up.
	
	
	What are the drivers that lead a portfolio manager to rebalance portfolio weights? In Part I, the case where time triggers a portfolio rebalancing will be explored. Although this Time-Driven approach is the most intuitive, it might incur in non-negligible transaction costs if the rebalancing frequency is high. On the other hand, in Part II, what causes the portfolio mix to be readjusted will be the fact that the risky asset cumulative return hits a lower or upper barrier. This portfolio rebalancing mechanics leads to the so-called Event-Driven approach to Asset Allocation.
	
	\vspace{0.5cm}
	\noindent
	\textbf{Keywords}: Asset Allocation, Stochastic Reachability, Time-Driven approach, Event-Driven approach.

\end{abstract}



\begin{otherlanguage}{italian}
	\begin{abstract}
		Contrariamente a quanto accade oggi per i maggiori investitori istituzionali, prima degli anni '50 chi investiva in borsa lo faceva senza adottare un metodo rigorosamente scientifico. Il cambio di rotta avvenne nel 1952, quando Harry Markowitz, fondatore della moderna teoria del portafoglio, pubblicò \textit{Portfolio Selection}, un articolo che aprì la strada ad un flusso di nuove idee provenienti dall'accademia e da svariate discipline scientifiche, idee che saranno destinate a rivoluzionare il modo di investire nei decenni successivi. 
		
		Proseguendo in questa direzione, nel nostro lavoro vengono utilizzati recenti risultati in Stochastic Reachability (concetto sviluppato nella teoria dei Sistemi di Controllo) per riadattarli in un contesto di Asset Allocation. In particolare, una volta individuato un adeguato profilo di rischio e un rendimento da raggiungere, il modello produrrà una strategia di investimento che massimizza la probabilità di raggiungere tale rendimento tenendo allo stesso tempo sotto controllo il rischio. Questa strategia avrà la caratteristica di essere \textit{Contrarian}, cioè prescrive di acquistare titoli rischiosi quando la performance del portafoglio è buona mentre prescriverà di venderli, in favore di titoli privi di rischio, se la performance è bassa.
		
		Con che criterio dunque, un gestore decide di ribilanciare i pesi di portafoglio? Nella prima parte del lavoro verrà presentato l'approccio Time-Driven, in cui è il tempo a dettare la riallocazione dei pesi (e.g. settimanalmente). Nonostante questo sia il metodo più intuitivo, ha come svantaggio gli elevati costi di transazione nel caso di frequenza di riallocazione alta. Nella seconda parte invece, l'approccio che si segue è quello Event-Driven. Ciò significa che una riallocazione di portafoglio viene effettuata solo quando il valore assoluto del rendimento cumulato dell'asset rischioso supera una certa soglia.
		
		
		\vspace{0.5cm}
		\noindent
		\textbf{Parole chiave}: Asset Allocation, Stochastic Reachability, Time-Driven approach, Event-Driven approach.
		
		
	\end{abstract}
\end{otherlanguage}

\tableofcontents
%----------------------------------------------------------------------------------------
%	THESIS CONTENT - CHAPTERS
%----------------------------------------------------------------------------------------

\mainmatter

\chapter{Introduction}

Since the first stock exchange opened for trades, investors have been trying to find ways to allocate their wealth among different securities in order to maximize returns. As a matter of fact, the techniques which are being employed nowadays in the investment industry are quite different from those used in the first half of the 20th century. "Fifty years ago, the business of managing other people's money was very much an art not a science, and was largely a matter of finding someone who was privy to inside information. But during the 1950s, 1960s and 1970s, academics changed the study of what became known as portfolio management. They did so in the face of much initial resistance and scepticism from the industry"\footnote{See \cite{Econ}.}. 

\section{The academia meets the industry}
In those days, any market participant would have been aware that investing was a risky business (nothing ventured, nothing gained). However, a formal and systematic connection between risk and return was still missing. The "annus mirabilis" in asset allocation was 1952, when Harry Markowitz published his pioneering paper \textit{Portfolio Selection} in the \textit{Journal of Finance}, starting the academic invasion of the financial industry. In his paper, which is considered to be the starting point of \gls{MPT}, Markowitz outlined for the first time how investors should allocate assets so as to achieve the highest returns given a certain level of risk. After having estimated expected returns and the covariance between each security, an investor, according to Markowitz, has to solve a quadratic programming problem for obtaining the so-called \textit{efficient portfolio frontier}. Any portfolio (a mix of securities) belonging to the frontier is efficient in the sense that it provides the highest expected return for a given level of risk (standard deviation). Moreover, each of these portfolios has the feature of being \text{deversified}. In Markowitz's words\footnote{See \cite{bernstein2012}.}:
\begin{quote}
	A portfolio with sixty different railway securities, for example, would not be as well diversified as the same size portfolio with some railroad, some public utility, mining, various sort of manufacturing, etc. The reason is that it is generally more likely for firms within the same industry to do poorly at the same time than for firms in dissimilar industries. 
\end{quote}

Building on Markowitz's work, the second major breakthrough is the \gls{CAPM} and it was made by another University professor, William Sharpe. The \gls{CAPM}, which appeared in the \textit{Journal of Finance} in 1964, allows one to compute the expected return from an asset in terms of its risk. The risk is divided into two components, namely a \textit{systematic risk} (which is related to the return of the whole market and cannot be eliminated) and a \textit{non-systematic risk} (which is unique to the asset and could be eliminated by diversifying the portfolio) \cite{hull2014}. In spite of the unrealistic assumptions which the \gls{CAPM} is based on \cite{hull2015risk},  it has proved to be a useful tool for portfolio managers. Other key contributions academia made to the industry are the \gls{EMH} (Fama, 1970) and the \gls{BS} model. \gls{EMH} is a theory that states that security prices perfectly reflect all available information in the market and, consequently, the whole market cannot be beaten. On the other hand, the \gls{BS} model is a mathematically-rigorous theory for pricing options.

Although all of the models above have shown some shortcomings when applied to the complex reality of capital markets, they have been crucial steps for reaching what Asset Allocation is today. Nonetheless, ideas from other scientific fields have continued to  enrich Finance and in particular Asset Allocation. An example of this flourishing contamination is the techniques of Stochastic Reachability presented in this works which have been borrowed from the theory of Control Systems and applied to the asset allocation problem.



\section{Structure}
In this section the structure of the thesis is outlined.

In Chapter \ref{chpt:Model_Description} we introduce the asset allocation problem and all the quantities related to it, such us the asset class return vector and the portfolio dynamics. Then, the concept of Stochastic Reachability is explored giving an idea about its possible fields of application. We conclude the chapter by discussing the mathematical formulation of the asset allocation problem in a Stochastic Reachability framework. In particular, the \gls{ODAA} algorithm, which all the thesis rely on, is enunciated.

In Chapter \ref{chpt:assetclass_returns} we discuss three models which could be employed to describe the probabilistic properties of the asset class returns vector, namely the \gls{G}, \gls{GM} and \gls{GH} model. For each of them, the model-related features (\textit{risk} constraint and portfolio value density function) of the \gls{ODAA} algorithm are obtained.

In Chapter \ref{chpt:calibration} we focus on techniques for calibrating the models presented in Chapter \ref{chpt:assetclass_returns} to market data. For the \gls{GM} case, calibration performance between three calibration methods are compared.


Chapter \ref{chpt:NumResTD} is dedicated to presenting the numerical results for the time-driven approach. The \gls{ODAA} asset allocation strategy is compared to the Constant-Mix and the CPPI, which are two benchmark policies in the industry. This will end Part \ref{part:1}

Part \ref{part:2} begins with Chapter \ref{chpt:ED}, where a brief introduction of the \gls{DES} theory is given. Afterwards, the asset allocation problem is cast in an \gls{ED} setting, calibration of model parameters is discussed and numerical results are given. 

Chapter \ref{chpt:Model_Extensions} constitutes the original part of the thesis. In this chapter we attempt to generalize the basic \gls{ED} model of Chapter \ref{chpt:ED} in two different ways: first, by modeling the risky asset as a \gls{GBM} and then by assuming a stochastic dynamics (Vasicek model) for the risk-free interest rate. 


Finally, in Chapter \ref{chpt:Conclusions} we sum up what has been achieved in this thesis and propose future research directions.

The MATLAB code used to implement the models presented in the thesis can be found in the following GitHub repository: \url{https://github.com/skiamu/Thesis}. 













\part{Time-driven approach}

\chapter{Model Description}\label{chpt:Model_Description}
\glsresetall
In this chapter, first the basic financial quantities are introduced and the asset allocation problem is stated, then the same problem will be embedded in a dynamical control system framework which will allows us to develop the stochastic reachability approach to portfolio construction. We closely follow \cite{Pola12},\cite{Pola06} and \cite{Pola}.
\section{Portfolio construction}
In the financial industry, a group of securities that exhibits similar characteristics in the market place and is subject to the same regulation is called \textbf{asset class}. Typical asset classes include stocks, bonds, real estate, cash and commodities. The discipline consisting in allocating investor's wealth among different asset classes is called \textbf{asset allocation}. We will now introduce the financial quantities and a formal mathematical setting suitable for describing the asset allocation problem. Let $(\Omega,\mathcal{F},\mathbb{P})$ be the underlying  probability space and consider a discrete set of time indexed by $k \in \mathbb{N}$. Moreover, let us consider a universe of $m \in \mathbb{N}$ asset classes. Asset classes' performance at period $k$ is described by a $m$-dimensional random vector 
$\bm{w}_k = \begin{bmatrix} w_k(1),\ldots,w_k(m)\end{bmatrix}^T $ 
where 
$$ w_k(i) = \frac{z_k(i)-z_{k-1}(i)}{z_{k-1}(i)}, \quad i = 1,\ldots,m$$
is the rate of return of the $i$-th asset class and $\{z_k(i)\}_{k \in \mathbb{N}}$ the $i$-th asset class price process. In general, the correlation of $\bm{w}_k$ can be of two kinds:
\begin{itemize}
	\item \textit{synchronous} correlation, that is the correlation among different asset class at the same time period (i.e. correlation between $w_k(i)$ and $w_k(j)$ for $i,j=1,\ldots,m$)
	\item \textit{time-lagged} correlation, that is the correlation among different asset class at different time period (i.e. correlation between $w_k(i)$ and $w_{k'}(j)$, with $k\neq k'$ for $i,j=1,\ldots,m$).
\end{itemize}
As the time-lagged correlation is usually negligible for short time period, $\bm{w}_k$ is a synchronous-correlated random vector. Standard notation is used for Expected Returns and Covariance Matrix 

\begin{align*}
\mu_k(i) & = \mathbb{E}\big[w_k(i)\big], \quad i = 1,\ldots,m \quad k \in \mathbb{N} \\[1.5ex]
\Sigma_k(i,j) &  = \mathbb{E}\bigg[\Big(w_k(i)-\mu_k(i)\Big)\Big(w_k(j)-\mu_k(j)\Big)\bigg], \quad i,j = 1,\ldots,m \quad k \in \mathbb{N}.
\end{align*}

 An asset allocation at period $k \in \mathbb{N}$ is a vector $\bm{u}_k \in \mathbb{R}^m$ whose $i$-th element indicates the percentage of wealth to be invested in asset class $i$. This vector is the leverage the asset manager has at his disposal for driving the portfolio value towards his goals. The portfolio performance over the period $[k-1,k]$ is measured by the portfolio return $$r_{k+1}=\frac{x_{k+1}-x_{k}}{x_k}$$ where $\{x_k\}_{k \in \mathbb{N}}$ is the portfolio value process. The portfolio return can also be expressed as a weighted average of each asset class return  as $$ r_{k+1} = \bm{u}_k^T \bm{w}_{k+1}.$$
 By combining the two previous relations we get the following recursive equation 
 \begin{equation}
 \boxed{x_{k+1} = x_k (1 + \bm{u}_k^T \bm{w}_{k+1})}
 \end{equation}
 which describes the time evolution of portfolio value. In plain words, the \textbf{asset allocation problem} consists in choosing the vector $\bm{u}_k$ at each time period $k \in \mathbb{N}$ (called \textbf{rebalancing time}) so as to achieve the  investor's goal. If the investor is mainly concerned about the final return, the allocation strategy is called \textbf{total-return allocation}. On the other hand, if his objective is beating a benchmark (index created to include multiple securities representing some aspect of the total market), the strategy is called \textbf{benchmark allocation}. In the following, we will consider only total-return portfolios.
 
 As well as setting the target return, the investor specifies other requirements that the portfolio manager must take into consideration. This means that  the asset allocation vector $\bm{u}_k$ is bound to stay within a feasible set $U_k$, for each $k \in \mathbb{N}$. In this work, the feasible set $U_k$ is obtained by imposing the following set of constraints
 \begin{itemize}
 	\item \textit{budget} constraint: $\sum_{i=1}^{m}u_k(i)=1$, all the wealth is invested in the portfolio
 	\item \textit{long-only} constraint: $u_k(i) \geq 0,\quad i = 1,\ldots,m$, no short-selling is allowed
 	\item \textit{risk} constraint: the metric value-at-risk ($V@R$) is used to limit portfolio risk. 
 \end{itemize}
The form of the \textit{risk} constraint will actually depend on the model used to describe the probabilistic properties of vector $\bm{w}_k$. In Chapter \ref{chpt:assetclass_returns} we will tackle this issue.
\section{Stochastic Reachability Approach}
In the previous section the financial setting has been laid, now it will be embedded in a more general framework by employing the theory of dynamical systems. We will see that this formalism will allow us to formulate the asset allocation problem as a \textbf{stochastic reachability} problem which will be solved by using \textbf{dynamic programming} (DP) techniques.
\subsection{The concept of Stochastic Reachability}
"In general terms, a reachability problem consists of determining if a given system trajectory will eventually enter a prespecified set starting from some initial state" \cite{Boj2012}. For deterministic systems, reachability analysis amounts to compute the set of states that can be reached by system trajectories. However, most of real-life problem are non-deterministic and uncertainty must be taken into account. In these cases, the main concern is determining the probability that the system reaches a prespecified set. "Typically, a certain part of the state space is "unsafe" and the control input of the system has to be chosen so as to keep the state away from it" \cite{Boj2012}. One of the most successful application of stochastic reachability techniques has been \gls{ATM}. "Within the \gls{ATM} context, safety-critical situation arise during flight when an aircraft comes closer than a minimum allowed distance to another aircraft or enters a forbidden region of the airspace. In the current \gls{ATM} system, air traffic controllers are in charge of guaranteeing safety by issuing to pilots corrective actions on their flight plans when a safety-critical situation is predicted" \cite{Boj2012}.

Conversely, when Stochastic Reachability is applied to the financial asset allocation problem, a dual viewpoint is taken. In this context, the focus is on driving the system state (the value of a portfolio of securities) into a "safe" set, and computing the probability that this occurs. The air traffic controller becomes a portfolio manager and signals issued to the pilot turns into orders to traders to buy or sell assets so as to adjust the portfolio mix.
\subsection{Mathematical Formulation}
 Let us introduce the following stochastic discrete-time dynamic control system
\begin{equation}
\label{eq:state_equation}
x_{k+1} = f(x_k,\bm{u}_k,\bm{w}_{k+1}) = x_k (1 + \bm{u}_k^T \bm{w}_{k+1}) 
\end{equation}
where, for any $k \in \mathbb{N}$
\begin{itemize}
	\item $x_k \in \mathcal{X} = \mathbb{R}$ is the system state, $\mathcal{X}$ the system space
	\item $\bm{u}_k \in U \subset \mathbb{R}^m$ is the control input, $U$ the control input space
	\item $\bm{w}_{k}$ is a $m$-dimensional random vector with density function $p_{\bm{w}_k}$
\end{itemize}

Let $\mathcal{U} = \big\{ \mu : \mathcal{X} \times \mathbb{N} \rightarrow U \big\}$ be the class of controls we focus on. Any $\mu \in \mathcal{U}$ is a map such that for any $x \in \mathcal{X}$ and any $k \in \mathbb{N}$, it associates an asset allocation vector $\bm{u}_k \in U$.
Given $N \in \mathbb{N}$ we define the set of control sequences as \[\mathcal{U}_N = \Big\{\pi = \{\mu_k\}_{k=0,\ldots,N}  : \mu_k \in \mathcal{U} \Big\}\] and call any $\pi \in \mathcal{U}_N$ a \textbf{control policy}. Moreover, let us denote by $\pi^k$ a control policy starting at period $k$, that is $\pi^k=\{\mu_k,\ldots,\mu_N\}$. We have all the ingredients to formulate the asset allocation problem in stochastic reachability terms.
\begin{problem}[Optimal Dynamic Asset Allocation]\label{prb:ODAA}
Given a finite time horizon $N \in \mathbb{N}$ and a sequence of target sets $\{X_1,\ldots,X_N \} $ such that each target set is a subset of the state space $\mathcal{X}$, find the optimal control policy $\pi^{\star} \in \mathcal{U}_{N-1}$ that maximizes the following objective function 
\begin{equation}\label{eq:obj_fun_ODAA}
\mathbb{P}\Big(\big\{\omega \in \Omega : x_0 \in X_0,\ldots,x_N \in X_N \big\} \Big).
\end{equation}
\end{problem}
The target sets $\{X_1,\ldots,X_N \}$ represent the investor's goal and we can think of them as the "good" states where we want the portfolio value to belongs to. For instance, a target set could be $X_k = [\underline{x}_k,\infty)$. Problem (\ref{prb:ODAA}) is going to be solved by resorting to dynamic programming. However, before doing that, we need to make explicit the dependence of (\ref{eq:obj_fun_ODAA}) on the control policy $\pi$. To this end, let $p_{f(x,\bm{u},\bm{w}_{k+1})}$ be the density of random variable (\ref{eq:state_equation}) and let us introduce the following function.
\begin{definition}[Value function]
	Given a sequence of target sets $\{X_k\}_{k\geq0}$, the \textbf{value function} associated with Problem \ref{prb:ODAA} is the following real map
	\begin{align*}
	V \colon \mathbb{N}\times \mathcal{X}\times \mathcal{U} & \rightarrow [0,1]\\
	(k,x,\pi^k) & \mapsto V(k,x,\pi^k)
	\end{align*}
	such that 
	\[V(k,x,\pi^k)=
	\begin{cases}
	    \mathbbm{1}_{X_N}(x) & \quad \text{if} \quad k = N \\
	    \int_{X_{k+1}}V(k+1,z,\pi^{k+1})p_{f(x,\bm{u},\bm{w}_{k+1})}(z)\mathrm{d}z & \quad \text{if} \quad k = N-1,\ldots,0.
	\end{cases}
	\]
	\end{definition}
It is now possible to link the objective function (\ref{eq:obj_fun_ODAA}) to the value function in the following way (see \cite{Pola}) \[\mathbb{P}\big(\{\omega \in \Omega : x_0 \in X_0,\ldots,x_N \in X_N \} \big) = V(0,x_0,\pi). \]
This result is extremely important since it allows us to rewrite the ODAA problem in terms of the value function as follows
\begin{problem}[Optimal Dynamic Asset Allocation 2]\label{prb:ODAA2}
  Given a finite time horizon $N \in \mathbb{N}$ and a sequence of target sets $\{X_1,\ldots,X_N \}$, find $$\pi^{\star} = \argmax_{\pi \in \mathcal{U}_{N-1}}V(0,x_0,\pi). $$	
\end{problem}
So far, we have reached an intermediate point where the ODAA problem has been restated in terms of a value function $V$. In this way, we can directly apply dynamic programming tools and solve it. The result is given in the following theorem, that is the cornerstone on which this work is based on.
\begin{theorem}[ODAA algorithm]\label{thm:rec_algo}
	the optimal value of the \gls{ODAA} Problem \ref{prb:ODAA2} is \[p^{\star} = J_0(x_0),\] where for any $x \in \mathcal{X},$ $J_0(x)$ is the final step of the following algorithm
	\begin{empheq}[box=\fbox]{align} \label{eq:rec_algo}
	J_N(x) & = \mathbbm{1}_{X_N}(x) \nonumber \\
	J_k(x) & = \sup_{\bm{u}_k \in U_k}\int_{X_{k+1}}J_{k+1}(z)p_{f(x,\bm{u}_k,\bm{w}_{k+1})}(z)\mathrm{dz} \\
	& k = N-1,\ldots,1,0. \nonumber
	\end{empheq}
\end{theorem}
The previous result provides us with a backward procedure (it starts at time $N$ and ends at time $0$) whose outputs are the optimal control policy $\pi^{\star}=\{\mu_0^{\star},\ldots,\mu_{N-1}^{\star}\}$ and the optimal joint probability $p^{\star}$ of reaching the target sets. It is worth pointing out some interesting features of algorithm in (\ref{eq:rec_algo}):
\begin{itemize}
	\item $J_k(x)$ is a function of portfolio realization $x \in \mathcal{X}$ at time $k$. This dependence is hidden behind the probability density function $p_{f(x,\bm{u}_k,\bm{w}_{k+1})}$
	\item the constrained optimization must be numerically carried out in a space ($U_k$) of dimension $m \in \mathbb{N}$. At each iteration $k = N-1,\ldots,1,0$, the optimization has to be repeated for each $x$ belonging to a suitable discretized grid
	\item the algorithm presented in theorem (\ref{thm:rec_algo}) does not depend on the particular distribution of random variable $f(x,\bm{u}_k,\bm{w}_{k+1})$. This fact gives us enough freedom to look outside the usual Guassian world
	\item given a period $k \in \mathbb{N}$ and a portfolio value realization $x \in \mathcal{X}$, $\mu_k^{\star}(x) \in U_k$ tells us which is the optimal allocation mix of our portfolio.
\end{itemize}

We now ask ourselves which probability distributions are suitable for vector $\bm{w}_{k+1}$; the answer to this question is the main objective of the next chapter.


\chapter{Asset Class Returns modeling}\label{chpt:assetclass_returns}
	
In this chapter, we address the asset class returns modeling issue. As it was noted above, the ODAA algorithm does not depend on a particular distribution for the asset class returns vector $\bm{w}_{k+1}$. However, by looking at equation  (\ref{eq:rec_algo}) we see that we need the explicit analytical form for the density function $p_{f(x,\bm{u}_k,\bm{w}_{k+1})}$. For this reason, we will be dealing exclusively with probability distributions closed under linear combination. In this work, we propose three such distributions:
\begin{itemize}
	\item Gaussian
	\item Gaussian Mixture
	\item Generelized Hyperbolic
\end{itemize}
For each of them, first we will give a theoretical introduction, secondly derive the portfolio value density function $p_{f(x,\bm{u}_k,\bm{w}_{k+1})}$ and an expression for the risk constraint (which depends on the distribution chosen).
\section{Gaussian model}
The first probability distribution we considered is the Gaussian 
\begin{definition}[Gaussian random vector]\label{def:gauss_rv}
	A $m$-dimensional random vector $\bm{w} = [w_1,\ldots,w_m]^T$ is \textbf{Gaussian} if every linear combination $\sum_{i=0}^{m}u_iw_i = \bm{u}^T \bm{w}$ has a one-dimensional Gaussian distribution.
\end{definition}
Let the asset class returns random vector $\bm{w}_{k+1}$ follow a Gaussian distribution with mean $\bm{\mu}$ and covariance matrix $\bm{\Sigma}$  $\big(\bm{w}_{k+1} \sim \mathcal{N}(\bm{\mu},\bm{\Sigma})\big)$. By definition we have \[x(1 + \bm{u}_k^T \bm{w}_{k+1}) \sim \mathcal{N}\Big(\underbrace{x(1 + \bm{u}_k^T \bm{\mu})}_{\tilde{\mu}}, \underbrace{x^2\bm{u}_k^T \Sigma \bm{u}_k}_{\tilde{\sigma}^2}\Big)\]
hence
\begin{equation}
\boxed{p_{f(x,\bm{u}_k,\bm{w}_{k+1})}(z) =  \frac{1}{\sqrt{2\pi}\tilde{\sigma}}\exp\big\{ -\frac{1}{2}\frac{(z-\tilde{\mu})^2}{\tilde{\sigma}^2}\big\}, \quad z \in \mathbb{R}.}
\end{equation}
 

Let us now introduce the two important concepts of loss function and value-at-risk that we will use to derive the portfolio risk constraint
\begin{definition}[loss function]\label{def:loss_function}
	Denoting the value of our portfolio at time $k \in \mathbb{N}$ by $x_{k+1}$, the \textbf{loss function} of the portfolio over the period $[k,k+1]$ is given by \[ L_{k+1}:= -\frac{(x_{k+1}-x_k)}{x_k}= -r_{k+1} = -\bm{u}_k^T \bm{w}_{k+1}.  \]
\end{definition}
\begin{definition}[Value-at-risk]
	Given some confidence level $1-\alpha \in (0,1) $ the \textbf{value-at-risk} ($V@R_{1-\alpha}$) of our portfolio is \[ V@R_{1-\alpha} = \inf\{l\in \mathbb{R} : \mathbb{P}\big(L_{k+1} \leq l \big) \geq 1-\alpha \}. \]
\end{definition}
	

The V@R is a risk measure commonly use by financial institutions to asses the risk they run to carry a portfolio for a specified period of time (the portfolio must be kept constant during this time period). For instance, if our portfolio has a $V@R_{0.99} = 7\%$, this means that with a confidence level of $99\%$ our portfolio does not suffer a loss greater or equal than $7\%$ over per period $[k,k+1]$ (e.g. a month). In our case, we receive the V@R specification ($V@R_{0.99} = 7\%$) in input by the investor (it is an indicator of its risk-aversion) and we will construct an asset allocation $\bm{u}_k$ that satisfies this risk constraint at each $k \in \mathbb{N}$.

Using definition (\ref{def:gauss_rv}) we have \[ L_{k+1} \sim \mathcal{N}\big(\underbrace{-\bm{u}_k^T \bm{\mu}}_{\mu_p},\underbrace{\bm{u}_k^T \bm{\Sigma} \bm{u}_k}_{\sigma^2_p} \big)\] therefore
\begin{align*}
\mathbb{P}\big(L_{k+1} \leq V@R_{1-\alpha}\big) 
& = \mathbb{P}\Big(Z \leq \frac{V@R_{1-\alpha} - \mu_p}{\sigma_p}  \Big) \\
& = 1 - \alpha
\end{align*}
\begin{equation}\label{eq:var_const_gaussian}
\implies \boxed{V@R_{1-\alpha} \geq -\bm{u}_k^T\bm{\mu} + z_{1-\alpha} \sqrt{\bm{u}_k^T \bm{\Sigma} \bm{u}_k}}
\end{equation}
where $Z$ is a standard normal random variable and $z_{1-\alpha}$ is the $1-\alpha$ quantile of the standard normal distribution. The \textit{risk} constraint in equation (\ref{eq:var_const_gaussian}), together with the \textit{budget} and \textit{long-only} constraint, define the control space $U_k$ which is the feasible set of the constrained optimization problem given in theorem (\ref{thm:rec_algo}).
\section{Gaussian Mixture model}
In this section we present the second asset class returns model, the \textbf{Gaussian Mixture model} (GM). After introducing the GM distribution we will derive the density and the \textit{risk} constraint, as we did for the Gaussian model. We closely follow \cite{BUCKLEY2008}.

The standard assumption that asset returns have a multivariate Gaussian distribution is a reasonable first approximation to reality and it usually has the big advantage of generating analytically tractable theories (e.g. Markowitz portfolio theory). However, the Gaussian model does not capture two important key features of asset returns which are observed in the market real data:
\begin{enumerate}
	\item the skewed (asymmetric around the mean) and leptokurtic (more fat-tailed than the Gaussian) nature of marginal probability density function
	\item the asymmetric correlation between asset returns, that is the tendency of volatilities and correlations to depend on the prevailing market conditions.
\end{enumerate}
To overcome this shortcomings, the Gaussian Mixture distribution is a validate alternative to the Gaussian one. Loosely speaking, the pdf of a GM random vector id a linear combination of Gaussian pdfs (called Gaussian regimes or mixing components). This closeness to the Normal distribution offers a good trade-off between analytical tractability and parsimony in the number of parameters. By adopting a GM model, it is possible to represent protuberances on the probability iso-density contour, as can be seen in the following figure
\begin{figure}[H]
	\caption{Example of a GM density contour plot with two mixing components.}
	\centering
	\includegraphics[width=0.7\textwidth]{Images/GMdensity.png}
\end{figure}
To obtain this highly non-linear dependence structure, we would usually need cross-moments of all order; a big advantage of the GM distribution is that its dependence structure is fully and conveniently captured by the means, covariance matrices and weights of each Gaussian regime (as we will see in the following).

We start the more formal introduction on the GM distribution giving its definition
\begin{definition}[GM distribution]
	An $m$-dimensional random vector $\bm{Z}$ has a \textbf{multivariate GM distribution} if its probability density function is of the form
	\[ p_{\bm{Z}}(\bm{z}) = \sum_{i=1}^{n} \lambda_i \varphi_{(\bm{\mu}_i,\bm{\Sigma}_i)}(\bm{z}), \quad \bm{z} \in \mathbb{R}^m, \] 
	where $\varphi_{(\bm{\mu}_i,\bm{\Sigma}_i)}$ is the multivariate Gaussian density with mean vector $\bm{\mu}_i$ and covariance matrix $\bm{\Sigma}_i$ and $\lambda_i$ are positive mixing weights summing to one.	
\end{definition}
The following proposition is crucial for our purposes since it tells us that linear combinations of GM random vector have a one-dimensional GM distribution
\begin{proposition}\label{prop:GM_lin_comb}
	Linear combinations of GM random vectors follow a univariate GM distribution. In particular, if $\bm{Z} \sim GM$ then $Y=\bm{\theta}^T \bm{Z}$,  $\forall \bm{\theta} \in \mathbb{R}^m$, has a GM distribution with probability density function \[ p_Y(y) = \sum_{i=1}^{n}\lambda_i \varphi_{(\mu_i,\sigma_i^2)}(y),\quad y \in \mathbb{R} \] where
	\[
	\begin{cases}
	\mu_i & = \bm{\theta}^T \bm{\mu}_i \quad i = 1,\ldots,m \\
	\sigma_i^2 & = \bm{\theta}^T \bm{\Sigma}_i \bm{\theta} \quad i = 1,\ldots,m
	\end{cases} \]
\end{proposition}
\begin{proof}
	The characteristic function (CF) of a GM random vector is the linear combination of the CF of the Gaussian mixing components. Indeed 
	\begin{align*}
	\phi_{\bm{Z}}(\bm{u}) &= \mathbb{E}[\exp\{i \bm{u}^T \bm{Z}\} ] = \int_{\mathbb{R}^m}\exp\{i \bm{u}^T \bm{z}\}p_{\bm{Z}}(\bm{z})\mathrm{d}\bm{z} = \\
	& = \int_{\mathbb{R}^m}\exp\{i \bm{u}^T \bm{z}\}\sum_{i=1}^{n}\lambda_i \varphi_{(\bm{\mu}_i,\bm{\Sigma}_i)}(\bm{z})\mathrm{d}\bm{z} = \\
	& = \sum_{i=1}^{n}\lambda_i \phi_{\bm{X}_i}(\bm{u}), \quad \bm{u} \in \mathbb{R}^m
	\end{align*}
	where $\bm{X}_i \sim \mathcal{N}\big(\bm{\mu}_i,\bm{\Sigma}_i\big)$. Therefore, $\forall \bm{\theta} \in \mathbb{R}^m$ we have
	\begin{align*}
	\phi_{\bm{\theta}^T \bm{Z}}(u) & = \mathbb{E}[\exp\{iu(\bm{\theta}^T\bm{Z})\}] = \mathbb{E}[\exp\{i(u\bm{\theta}^T)\bm{Z}\}] = 
    \phi_{\bm{Z}}(u\bm{\theta}) = \\
    & = \sum_{i=1}^{n}\lambda_i \phi_{X_i}(u\bm{\theta}) = \sum_{i=1}^{n}\lambda_i \exp\{iu\underbrace{\bm{\theta}^T\bm{\mu}_i}_{\mu_i}-\frac{1}{2}u^2\underbrace{\bm{\theta}^T\bm{\Sigma}_i\bm{\theta}}_{\sigma_i^2}\} = \\
	& = \sum_{i=1}^{n}\lambda_i \phi_{\widetilde{X}_i}(u), \quad u \in \mathbb{R}
	\end{align*}
	where $\widetilde{X}_i \sim \mathcal{N}\big(\mu_i,\sigma_i^2 \big)$. Since the CF completely characterizes the distribution (see theorem 14.1 in \cite{jacod2000probability}) we have the result.
\end{proof}
Suppose the asset class returns vector\footnote{for the sake of clarity we drop the subscript $k+1$ when it is not needed} $\bm{w}$ follow a Gaussian Mixture distribution ($\bm{w} \sim GM $). We want to compute the density of random variable $f(x,\bm{u},\bm{w}) = x(1 + \bm{u}^T\bm{w})$. Thanks to proposition (\ref{prop:GM_lin_comb}) we know that the random variable $\bm{u}^T\bm{w}$ follows itself a GM (univariate) distribution. Moreover, by integration we easily derive its cumulative distribution function (CDF) allowing us to write
\begin{align*}
F_{f(x,\bm{u},\bm{w})}(z) & = \mathbb{P}\big(x(1+\bm{u}^T\bm{w})\leq z \big) = F_{x\bm{u}^T\bm{w}}(z-x)\\
& = \sum_{i=1}^{n}\lambda_i \Phi\Big(\frac{z - x(1+\bm{u}^T\bm{\mu}_i)}{\sqrt{x^2\bm{u}^T\bm{\Sigma}_i\bm{u}}}\Big), \quad z \in \mathbb{R}
\end{align*}
where $\Phi$ is the standard normal CDF. Differentiating with respect to $z$, we have
\begin{equation}
\boxed{p_{f(x,\bm{u},\bm{w})}(z) = \sum_{i=1}^{n}\lambda_i \varphi_{(\mu_i,\sigma_i^2)}(z), \quad z \in \mathbb{R}}
\end{equation}
where 
\[
\begin{cases}
	\mu_i &= x(1+\bm{u}^T \bm{\mu}_i) \\
	\sigma_i^2 & = x^2\bm{u}^T\bm{\Sigma}_i \bm{u}.
\end{cases}
\]


We now turn to the problem of computing the \textit{risk} constraint under the GM distribution assumption. We will follow two different approaches. Suppose we are given the $V@R_{1-\alpha}$ specification (e.g. $7\%$); by using definition (\ref{def:loss_function}) we have \[ \mathbb{P}\big(L \leq V@R_{1-\alpha} \big) = F_L(V@R_{1-\alpha}) \geq 1-\alpha \] as noted above, the CDF of $L = -\bm{u}^T \bm{w}$ is known, therefore \[\sum_{i=1}^{n}\lambda_i \Phi\Big(\frac{V@R_{1-\alpha} - \mu_i}{\sigma_i}\Big) \geq 1-\alpha \quad \implies \quad\]
\begin{equation}
 \boxed{\sum_{i=1}^{n}\lambda_i \Phi\Big(-\Big\{\frac{V@R_{1-\alpha} - \mu_i}{\sigma_i} \Big\} \Big) \leq \alpha}  
\end{equation}
where \[
\begin{cases}
\mu_i & = -\bm{u}^T \bm{\mu}_i \\
\sigma_i^2 & = \bm{u}^T \bm{\Sigma}_i \bm{u}.
\end{cases}\]
We present also an alternative method to limit the risk exposure of our portfolio which turns out to be less computationally intensive. The idea is to set an upper bound to portfolio return volatility in the following way 
\begin{equation}\label{eq:risk_upperbnd}
(\Var{r_{k+1}})^{\frac{1}{2}} = (\bm{u}_k^T \bm{\Lambda} \bm{u}_k)^{\frac{1}{2}} \leq \sigma_{max}
\end{equation}
where $\bm{\Lambda}$ is the covariance matrix of vector $\bm{w}_{k+1}$. Two questions are left open: how to compute $\bm{\Lambda}$ and how to link the upper bound $\sigma_{max}$ to the $V@R_{1-\alpha}$ specification given in input by the investor. As far as the former is concerned, the following proposition gives us the answer
\begin{proposition}
	The covariance matrix of a random vector with the GM distribution can be expressed in terms of mean vectors, covariance matrices and weights of the mixing components in the following way
	\[ \bm{\Lambda} = \sum_{i=1}^{n}\lambda_i\bm{\Sigma}_i + \sum_{i=1,j<i}^{n,n} \lambda_i\lambda_j(\bm{\mu}_i-\bm{\mu}_j)(\bm{\mu}_i-\bm{\mu}_j)^T.\]
\end{proposition}
To answer the latter, we use a Guassian approximation and the fact that if the rebalancing frequency is relatively small (e.g. weekly) the portfolio return mean is negligible. In the end, the obtain
\[ \sigma_{max} = \frac{V@R_{1-\alpha}}{z_{1-\alpha}}. \] 
\section{Generelized Hyperbolic model}
The last distribution we propose for our asset class returns modeling purposes is the Generalized Hyperbolic (GH). Like the GM, in its general form also the GH presents a non-elliptical behaviour with asymmetric and fat-tailed marginals. We proceed to give the formal definition and then derive the density of $f(x,\bm{u}_k,\bm{w}_{k+1})$
and the \textit{risk} constraint.
\begin{definition}[GH distribution]\label{def:GH}
	A $m$-dimensional random vector $\bm{X}$ is said to follow a \textbf{multivariate GH distribution} ($\bm{X} \sim GM_m(\lambda,\chi,\psi,\bm{\mu},\bm{\Sigma},\bm{\gamma})$) if \[ \bm{X} = \bm{\mu}+W\bm{\gamma}+\sqrt{W}A\bm{Z} \] where 
	\begin{itemize}
		\item $\bm{Z} \sim \mathcal{N}\big(\bm{0},I_d\big)$
		\item $A \in \mathbb{R}^{m \times d} $ is the Cholesky factor of dispersion matrix $\bm{\Sigma}$ ($A^TA = \bm{\Sigma}$)
		\item $\bm{\mu}, \bm{\gamma} \in \mathbb{R}^m$
		\item $W \sim \mathcal{N}^-(\lambda,\chi,\psi)$, $W \geq 0$ and $W \perp \bm{Z}$ (see Appendix \ref{app:B} for the definition of the GIG distribution). $W$ is sometimes called mixing random variable.
	\end{itemize}
\end{definition}
\begin{remark}
	\begin{itemize}
		\item $\lambda,\chi,\psi$ are shape parameters; the larger these parameters the closer the distribution is to the Gaussian
		\item $\bm{\gamma}$ is the skewness parameter. If $\bm{\gamma}= \bm{0}$ the distribution is symmetric around the mean
		\item $\bm{X}\lvert W = w \sim \mathcal{N}\big(\bm{\mu}+w\bm{\gamma},w\bm{\Sigma}\big)$.
	\end{itemize}
\end{remark}
The GH distribution contains some special cases:
\begin{itemize}
	\item If $\lambda=\frac{m+1}{2}$ we have a \textit{Hyperbolic} distribution
	\item If $\lambda=-\frac{1}{2}$ the distribution is called \textit{Normal Inverse Gaussian} (NIG)
	\item If $\chi = 0$ and $\lambda > 0$ we have the limiting case of the \textit{Variance Gamma} (VG) distribution
	\item If $\psi = 0$ and $\lambda < 0$ the resulting distribution is called \textit{Student-t}.
\end{itemize}
	
The following proposition gives us the closeness under linear transformation that we need for our modeling purposes
\begin{proposition}
	If $\bm{X} \sim GH_m(\lambda,\chi,\psi,\bm{\mu},\bm{\Sigma},\bm{\gamma})$ and $\bm{Y}=B\bm{X}+\bm{b}$, where $B \in \mathbb{R}^{d\times m}$ and $\bm{b} \in \mathbb{R}^d$, then
	\[ \bm{Y} \sim GH_d(\lambda,\chi,\psi,B\bm{\mu}+b,B\bm{\Sigma}B^T,B\bm{\gamma}). \]
\end{proposition}
Suppose $\bm{w}_{k+1} \sim GH_m(\lambda,\chi,\psi,\bm{\mu},\bm{\Sigma},\bm{\gamma})$. Applying the previous result to our case $\big(Y = f(x,\bm{u}_k, \bm{w}_{k+1}),\quad B=x\bm{u}_k^T,\quad b = x\big)$ we have \[ f(x,\bm{u}_k,\bm{w}_{k+1}) \sim GH_1(\lambda,\chi,\psi,\underbrace{x(1+\bm{u}_k^T\bm{\mu})}_{\widetilde{\mu}},\underbrace{x^2\bm{u}_k^T\bm{\Sigma}\bm{u}_k}_{\widetilde{\Sigma}},\underbrace{x\bm{u}_k^T\bm{\gamma}}_{\widetilde{\gamma}}) \] and the density reads as (see Appendix  \ref{app:B})
\begin{equation}\label{eq:GHdensity}
\boxed{p_{f(x,\bm{u}_k, \bm{w}_{k+1})}(z) = c \frac{K_{\lambda-\frac{1}{2}}\Big(\sqrt{\big(\chi+\widetilde{Q}(z) \big)\big(\psi+\widetilde{\gamma}^2/\widetilde{\Sigma} \big)\exp\big\{(z-\widetilde{\mu})\widetilde{\gamma}/\widetilde{\Sigma} \big\}}\Big)}{\Big(\sqrt{\big(\chi+\widetilde{Q}(z) \big)\big(\psi+\widetilde{\gamma}^2/\widetilde{\Sigma} \big)}\Big)^{\frac{1}{2}-\lambda}}}
\end{equation}
where \[ c = \frac{\big(\sqrt{\chi\psi}\big)^{-\lambda}\psi^{\lambda}\big(\psi+\widetilde{\gamma}^2/\widetilde{\Sigma}\big)^{\frac{1}{2}-\lambda}}{(2\pi\widetilde{\Sigma})^{\frac{1}{2}}K_{\lambda}(\sqrt{\chi\psi})} \]
and $\widetilde{Q}(z)= (z-\widetilde{\mu})/\widetilde{\Sigma}$.


As far as the \textit{risk} constraint is concerned, we adopt here the alternative approach expressed in Equation (\ref{eq:risk_upperbnd}). The covariance matrix $\bm{\Lambda}$ is easily derived from the definition of a GH random vector (Definition \ref{def:GH}) and Equation (\ref{eq:GIG_moment}); in the end we obtain
\begin{equation}
\bm{\Lambda} = \Var{W}\bm{\gamma}\bm{\gamma}^T+\mathbb{E}[W]\bm{\Sigma}
\end{equation}
where
\begin{align*}
\mathbb{E}[W] & = \Big(\frac{\chi}{\psi}\Big)^{\frac{1}{2}}\frac{K_{\lambda+1}(\sqrt{\chi\psi})}{K_{\lambda}(\sqrt{\chi\psi})}\\
\Var{W} & = \Big(\frac{\chi}{\psi}\Big)\frac{1}{K_{\lambda}(\sqrt{\chi\psi})}\Big\{K_{\lambda+2}(\sqrt{\chi\psi})-\frac{K_{\lambda+1}^2(\sqrt{\chi\psi})}{K_{\lambda}(\sqrt{\chi\psi})} \Big\}.
\end{align*}





\chapter{Model Calibration}
In this chapter we show how to calibrate the models introduced in Chapter \ref{chpt:assetclass_returns} to market data. The asset class menu we will consider consists in equity, bond and cash and a suitable index will be used to represent each of these markets (the dataset is discussed in Section \ref{}). We focus our attention only on GM and GH since calibrating the Gaussian model is trivial (it amounts to compute the sample mean and covariance matrix). As far as the GM model is concerned, we set the number of mixing Gaussian components to 2. In financial terms, the two mixing components could be interpreted as economic regimes, namely a \textit{tranquil} regime and a \text{distressed} one (see \cite{Brey2013}). Different calibration methods are available, namely the \textbf{Method of Moments} (MM), \textbf{Maximum Likelihood} (ML) estimation  and the \textbf{Expectation-Maximization} (EM) algorithm. Each of them will be discussed in Section \ref{sec:GM_calibration} and also a comparison between them will be provided. Finally, in Section \ref{sec:GH_calibration} the GH model will be fitted to data using the multi-cycle expectation conditional estimation (MCECM) algorithm. 
\section{GM calibration} \label{sec:GM_calibration}
The problem of estimating the parameters of a Gaussian Mixture distribution dates back to \cite{Pearson1894} and still nowadays it raises in a wide spectrum of different disciplines (Finance and Classification just to name a few). Thanks to the computational power available today, the EM algorithm is considered to be the state-of-the-art method for fitting the GM distribution. Nevertheless, MM and ML are worth studying as they shed light on different aspect of the problem at hand and they could provide the starting point for the EM algorithm. The main reference for the MM method is \cite{Everitt81}, for the ML \cite{casella2002} and for EM \cite{McNeil2005}.
\subsection{Method of Moments}\label{subsec:MM}
In this subsection we present the Method of Moments for calibrating a 3-dimensional Gaussian Mixture distribution with $n=2$ mixing component. The idea behind MM is to match observed and theoretical moments; this translates into a system of polynomial equations that most of the times, for real-size problems, has to be solved numerically. Since we need to fit a 3-dimensional distribution, we will work component-wise: moment-matching equations will be written  for each component together with unimodality on each marginal. In order to keep the number of parameters to a reasonable degree, we will suppose a common correlation matrix between the two Gaussian mixing components.

Let $\{\bm{X}_1,\ldots,\bm{X}_n \}$ be a random sample from a GM distribution whose density function is 
\begin{equation}\label{eq:calibration_density}
f(\bm{z}) = \lambda \varphi_{(\bm{\mu}_1,\bm{\Sigma}_1)}(\bm{z}) + (1-\lambda)\varphi_{(\bm{\mu}_2,\bm{\Sigma}_2)}(\bm{z}),\quad \bm{z} \in \mathbb{R}^3
\end{equation}
Our goal is to estimate $\{\lambda,\bm{\mu}_1,\bm{\Sigma}_1,\bm{\mu}_2,\bm{\Sigma}_2 \}$ from the random sample. Due to the assumption of a shared correlation matrix, the number of actual parameters to estimate is 16: $\lambda$, 6 means, 6 standard deviations and 3 correlations.
To set the notation we give the following definition
\begin{definition}[theoretical and sample moments]
	Let $X$ be a random variable and $\{x_1,\ldots,x_n\} $ a realization of a random sample. The first four theoretical and sample moments are:
	\begin{align*}
	\mu_X & = \mathbb{E}[X] \quad & \bar{x} &= \frac{1}{n}\sum_{j=1}^{n}x_j\\
	\sigma^2_X & = \mathbb{E}\big[(X-\mu_X)^2\big] \quad & s^2 &= \frac{1}{n}\sum_{j=1}^{n}(x_j-\bar{x})^2\\
	\gamma_X & = \frac{1}{\sigma_X^3}\mathbb{E}\big[(X-\mu_X)^3\big] \quad & \widehat{\gamma} & = \frac{\frac{1}{n}\sum_{j=1}^{n}(x_j-\bar{x})^3}{\Big(\sqrt{\frac{1}{n}\sum_{j=1}^{n}(x_j-\bar{x})^2} \Big)^3}\\
	\kappa_X & = \frac{1}{\sigma_X^4}\mathbb{E}\big[ (X-\mu_X)^4\big] \quad & \widehat{\kappa}&= \frac{\frac{1}{n}\sum_{j=1}^{n}(x_j-\bar{x})^4}{s^4}
	\end{align*}
\end{definition}
Let $\bm{X}$ be a random vector with density (\ref{eq:calibration_density}), its $i$-th marginal is \[f_{X_i}(z) = \varphi_{(\mu_{1i},\sigma^2_{1i})}(z)+(1-\lambda)\varphi_{(\mu_{2i},\sigma^2_{2i})}(z), \quad z \in \mathbb{R} \quad  i \in \{1,2,3\} \]
where $\mu_{ji}$ and $\sigma^2_{ji}$ denote respectively the $j$-th element of the $i$-th mixing component mean vector  and the $j$-th diagonal entry of the $i$-th mixing component covariance matrix, $i \in \{1,2,3\}$, $j \in \{1,2\}$ (namely the first subscripts indicates the dimension, the second the mixing component). Computing explicitly the theoretical moments we obtain
\begin{align*}
\mu_{X_i} & = \lambda\mu_{1i}+(1-\lambda)\mu_{2i}\\[15pt] 
\sigma^2_{X_i} & = \lambda(\sigma^2_{1i}+\mu^2_{1i})+(1-\lambda)(\sigma^2_{2i}+\mu^2_{2i})\\[15pt]
\gamma_{X_i} & = \frac{1}{\sigma^3_{X_i}}\Big\{\big[\lambda(\mu^3_{1i}+3\mu_{1i}\sigma^2_{1i}) + (1-\lambda)(\mu^3_{2i}+3\mu_{2i}\sigma^2_{2i})\big] -3\mu_{X_i}\sigma^2_{X_i}-\mu^3_{X_i} \Big\}\\[15pt]
\kappa_{X_i}&= \frac{1}{\sigma^4_{X_i}}\Big\{\big[\lambda(\mu^4_{1i}+6\mu^2_{1i}\sigma^2_{1i}+3\sigma^4_{1i})+(1-\lambda)(\mu^4_{2i}+6\mu^2_{2i}\sigma^2_{2i}+3\sigma^4_{2i}) \big] +\\
& \quad -\mu^4_{X_i} -6\mu^2_{X_i}\sigma^2_{X_i}-4\gamma_{X_i}\sigma^3_{X_i}\mu_{X_i} \Big\}
\end{align*}
where $i \in \{1,2,3\}$. Equating them with their sample counterparts gives us the first twelve moment equations. The three correlation equations are derived equating the theoretical covariances (written as a function of correlation coefficients $\rho_{ij}$)
\[ \sigma_{X_iX_j}= \lambda\rho_{ij}\sigma_{1i}\sigma_{1j}+(1-\lambda)\rho_{ij}\sigma_{2i}\sigma_{2j}+\lambda(1-\lambda)(\mu_{1i}-\mu_{2i})(\mu_{1j}-\mu_{2j})   \] and the sample ones \[ \widehat{\sigma}_{X_iX_j} = \frac{1}{n}\sum_{s=1,t=1}^{n,n}(x_s-\bar{x})(x_t-\bar{x})
\]
 $i \in \{1,2,3\} \quad j<i.$ So far, we have derived 15 equations in 16 unknown parameters. In order to have as many equations as unknown parameters, we solve the moment equation system by numerically minimizing the sum of square differences between theoretical and sample moments for different values of $\lambda$ in a discretized grid of the interval $[0,1]$. The optimal $\lambda$ will the one giving the smallest residual. Moreover, in the optimization process we also imposed the following uni-modality constraints on each marginal\footnote{see \cite{Eisenberg64} for the proof of this sufficient condition for uni-modality for a 2-mixing-component GM density}
 \[ (\mu_{2i} - \mu_{1i})^2 \leq \frac{27}{4}(\sigma^2_{2i}\sigma^2_{1i})/(\sigma^2_{1i}+\sigma^2_{2i}) \quad i \in \{1,2,3\}\]
 and positive-definiteness constraints on the standard deviation and correlation parameters. The uni-modality constraint is required  since bi-modal return distributions are not observed in the market.

\subsection{Expectation-Maximization}
In this section we introduce the EM algorithm for calibrating a GM model. Before diving into it, we need to define the maximum-likelihood estimator since the EM algorithm comes into play to solve difficulties in the ML method.
\begin{definition}[Likelihood function]
	Let $\bm{x}=\{x_1,\ldots,x_N\}$ be a realization of a random sample from a population with pdf $f(x\lvert\bm{\theta})$ parametrized by $\bm{\theta}=[\theta_1,\ldots,\theta_k]^T$. The \textbf{likelihood function} is defined by \[ L(\bm{\theta}\lvert\bm{x}) = L(\theta_1,\ldots,\theta_N\lvert x_1,\ldots,x_k) = \prod_{i=1}^{N}f(x_i\lvert\bm{\theta}). \]	
\end{definition}
The following definition of a maximum likelihood estimator is taken from \cite{casella2002}
\begin{definition}[maximum-likelihood estimator]
	For each sample point $\bm{x}$, let $\widehat{\bm{\theta}}(\bm{x})$ be the parameters value at which $L(\bm{\theta}\lvert\bm{x})$  attains its maximum as a function of $\bm{\theta}$, with $\bm{x}$ held fixed. A \textbf{maximum-likelihood estimator} (MLE) of the parameters vector $\bm{\theta}$ based on a random sample $\bm{X}$ is $\widehat{\bm{\theta}}(\bm{X})$
\end{definition}
Intuitively, the MLE is a reasonable estimator since is the parameter point for which the observed sample is most likely. However, its main drawback is that finding the maximum of the likelihood function (or its logarithmic transformation) might be difficult both analytically and numerically. Consequently, the idea is to adopt an iterative procedure that converges to a local maximum.
In order to focus on the idea behind the EM algorithm and not on technical details, we will present it in the simpler case of a uni-variate GM distribution with 2 mixing components (as presented in \cite{hastie2009}). The interested reader can refer to \cite{hastie2009} for the general case or \cite{Plasse2013} for a more throughout discussion.

Consider a mixture of two Gaussian random variables
\[X = (1-\Delta)X_1 + \Delta X_2 \] where $X_1 \sim \mathcal{N}\big(\mu_1,\sigma^2_1\big)$, $X_2 \sim \mathcal{N}\big(\mu_2,\sigma^2_2\big)$ and $\Delta \sim B(\lambda)$ is the mixing random variable. The density function of $X$, parametrized by $\bm{\theta} = [\lambda,\mu_1,\sigma^2_1,\mu_2,\sigma^2_2]^T$, is
\[f_X(x) = (1-\lambda)\varphi_{(\mu_1,\sigma^2_1)}(x)+\lambda \varphi_{(\mu_2,\sigma^2_2)}(x), \quad x \in \mathbb{R}. \] Our objective is to find an estimate $\widehat{\bm{\theta}}$ of $\bm{\theta}$. Let $\bm{x} = \{x_1,\ldots,x_N\}$ be a realization of a random sample (our data at hand), the log-likelihood function is
\begin{equation}\label{eq:log-likelihood}
l(\bm{\theta};\bm{x}) = \sum_{i=1}^{N}\log\big[(1-\lambda)\varphi_{(\mu_1,\sigma^2_1)}(x_i)+\lambda\varphi_{(\mu_2,\sigma^2_2)}(x_i) \big]
\end{equation}
In higher dimensions, the direct maximization of (\ref{eq:log-likelihood}) is difficult and prevent the ML method from being successful. Let us suppose to know the following latent random variables
\[ \Delta_i = 
\begin{cases*}
1 \quad  \text{if $X_i$ comes from model 2}\\
0 \quad \text{if $X_i$ comes from model 1}
\end{cases*}
\]
for $i = 1,\ldots,N$. Model 1 or 2 is intended the population whose density is the first or second Gaussian component. In this hypothetical case, the log-likelihood function would be 
\begin{align*}
l_0(\bm{\theta};\bm{x},\bm{\Delta}) & = \sum_{i=1}^{N}\big[(1-\Delta_i)\log\big(\varphi_{(\mu_1,\sigma^2_1)}(x_i)\big)+\Delta_i\log\big(\varphi_{(\mu_2,\sigma^2_2)}(x_i)\big)\big]+\\
& \quad + \sum_{i=1}^{N}\big[(1-\Delta_i)\log(1-\lambda)+\Delta_i \log(\lambda) \big].
\end{align*}
If the $\Delta_i$'s were known, the maximum-likelihood estimate for $\mu_1$ and $\sigma^2_1$ would be the sample mean and sample variance from the observations with $\Delta_i = 0$. The same holds true for $\mu_2,\sigma^2_2$ and $\Delta_i=1$. The estimate for $\lambda$ would be the proportion of $\Delta_i = 1$. However, as the 
$\Delta_i$'s are not known, we use as their surrogates the conditional expectations \[ \gamma_i(\bm{\theta}) = \mathbb{E}[\Delta_i \lvert \bm{\theta},\bm{x}]= \mathbb{P}\big(\Delta_i= 1 \lvert\bm{\theta},\bm{x} \big) \quad i = 1,\ldots,N \] called \textit{responsability} of model 2 for observation $i$. The iterative procedure called EM algorithm consists in alternating an \textit{expectation} step in which we assign to each observation the probability to come from each model, and a \textit{maximization} step where these responsabilities are used to update ML estimates.
\begin{algorithm}[H]
	\caption{Expectation-Maximization (EM) for 2-component GM}
	\begin{algorithmic}[1]
		\State take initial guesses for parameters $\widehat{\mu_1},\widehat{\mu_2},\widehat{\sigma}^2_1,\widehat{\sigma}^2_2, \widehat{\lambda}$
		\State\label{state:2} \textit{Expectation} step:  compute responsabilities
		\[ \widehat{\gamma}_i = \dfrac{\widehat{\lambda}\varphi_{(\widehat{\mu}_2,\widehat{\sigma}^2_2)}(x_i)}{(1-\widehat{\lambda})\varphi_{(\widehat{\mu}_1,\widehat{\sigma}^2_1)}(x_i) + \widehat{\lambda}\varphi_{(\widehat{\mu}_2,\widehat{\sigma}^2_2)}(x_i)},\quad i=1,\ldots,N  \]
		\State\label{state:3} \textit{Maximization} step: compute weighted means and standard deviations
		\begin{align*}
		\widehat{\mu}_1 & = \frac{\sum_{i=1}^{N}(1-\widehat{\gamma}_i)x_i}{\sum_{i=1}^{N}(1-\widehat{\gamma}_i)}, \qquad & \widehat{\sigma}^2_1 & = \frac{\sum_{i=1}^{N}(1-\widehat{\gamma}_i)(x_i-\widehat{\mu}_1)^2}{\sum_{i=1}^{N}(1-\widehat{\gamma}_i)}\\
		\widehat{\mu}_2 & = \frac{\sum_{i=1}^{N}\widehat{\gamma}_ix_i}{\sum_{i=1}^{N}\widehat{\gamma}_i}, \qquad & \widehat{\sigma}^2_2 & = \frac{\sum_{i=1}^{N}\widehat{\gamma}_i(x_i-\widehat{\mu}_2)^2}{\sum_{i=1}^{N}\widehat{\gamma}_i}
		\end{align*}
		\State Iterate \ref{state:2} and \ref{state:3} until convergence.
	\end{algorithmic}
\end{algorithm}
A reasonable starting value for $\widehat{\mu}_1$ and $\widehat{\mu}_2$ is a random sample point $x_i$, both $\widehat{\sigma}_1, \widehat{\sigma}_2$ can be set equal to the sample variance and $\widehat{\lambda} = 0.5$. A full implementation of the EM algorithm is available in MATLAB.

\subsection{MM vs ML vs EM}
In this subsection we put the calibration methods into practice to see which one is better at recovering the parameters of a GM distribution. To this end, we simulated $10^4$ observations from a GM distribution with the following parameters
\begin{align*}
\bm{\mu}_1 & = 
\begin{bmatrix}
\num{6.11e-4} &  \num{1.373e-3} &  \num{2.34e-3}
\end{bmatrix}
\quad & \bm{\Sigma}_1 &= 
\begin{bmatrix}
\num{4.761e-9} & \num{2.474e-8} & \num{2.731e-8} \\
               & \num{3.21e-5}  & \num{-2.55e-6} \\
               &                & \num{3.656e-4}
\end{bmatrix} \\
\bm{\mu}_2 & = \begin{bmatrix}
\num{6.83e-4} &  \num{-1.61e-2} &  \num{-1.75e-2}
\end{bmatrix}
\quad & \bm{\Sigma}_2 &= 
\begin{bmatrix}
\num{3.844e-9} & \num{2.42e-8} & \num{6.739e-8} \\
               & \num{3.804e-5}  & \num{-7.644e-6} \\
               &                & \num{2.757e-3}
\end{bmatrix}
\end{align*}
and $\lambda = 0.98$. In order to have a fair comparison, the two Gaussian regimes have a common correlation matrix \[ R = 
\begin{bmatrix}
\num{1} & \num{6.33e-2} & \num{2.07e-8} \\
        & \num{1}       & \num{-2.36e-2} \\
        &                & \num{1}
\end{bmatrix}\]

The result is summarized in the following tables

% first table
\begin{longtable}{@{}lcccccc@{}} \toprule
	
	Parameter & MM & $e_{MM}$ (\%) & ML & $e_{ML}$ (\%) & EM & $e_{EM}$ (\%) \\ \midrule
	$\widehat{\mu}_{1}$  & $\num{6.167e-4}$ & $0.94$ &  & & $\num{6.11e-4}$ & 0.0264 \\ 
	\addlinespace[0.5em]
	$\widehat{\mu}_{2}$  & $\num{1.578e-3}$ & $14.98$ &  & & $\num{1.368e-3}$ & 0.366\\
	\addlinespace[0.5em]
	$\widehat{\mu}_{3}$  & $\num{2.396e-3}$ & $2.40$ &  & & $\num{2.174e-3}$ & 7.066\\
	\addlinespace[0.5em]
	$\widehat{\Sigma}_{11}$ &  $\num{2.757e-8}$ & $479.2$ &  & & $\num{4.704e-9}$ & 1.189   \\
	\addlinespace[0.5em]
	$\widehat{\Sigma}_{22}$ & $\num{3.215e-5}$ & $0.14$ &  & & $\num{3.155e-5}$ & 1.699   \\
	\addlinespace[0.5em]
	$\widehat{\Sigma}_{33}$ & $\num{3.092e-4}$  & $15.4$ &   & & $\num{3.661e-4}$ & 0.146  \\
	\addlinespace[0.5em]
	$\widehat{\Sigma}_{12}$ & $\num{-3.554e-8}$ & $243.6$  &  & & $\num{2.249e-8}$ & 9.123 \\
	\addlinespace[0.5em]
	$\widehat{\Sigma}_{13}$ & $\num{-3.1e-8}$  & $213.5$  &  & & $\num{3.487e-8}$ & 27.68  \\
	\addlinespace[0.5em]
	$\widehat{\Sigma}_{23}$ & $\num{-4.805e-7}$ & $81.20$  &  & & $\num{-2.756e-4}$ & 7.788  \\\bottomrule
	\addlinespace[0.5em]
	\caption{estimates for the first mixing component and respective estimation errors}
\end{longtable}
% second table
\begin{longtable}{@{}lcccccr@{}} \toprule
	Parameter & MM & $e_{MM}$ (\%) & ML & $e_{ML}$ (\%) & EM & $e_{EM}$ (\%) \\ \midrule
	$\widehat{\mu}_{1}$  & $\num{5.461e-4}$  & 20.03 & & & $\num{6.843e-4}$ & 0.193 \\ \addlinespace[0.5em]
	$\widehat{\mu}_{2}$  & $\num{-7.26e-3}$     & 54.91 & & & $\num{-1.554e-2}$ & 3.481\\
	\addlinespace[0.5em]
	$\widehat{\mu}_{3}$  & $\num{-8.11e-3}$   & 53.65 & & & $\num{-1.953e-2}$ & 11.605 \\
	\addlinespace[0.5em]
	$\widehat{\Sigma}_{11}$ & $\num{1.157e-8}$ & 201.07 & & & $\num{3.223e-9}$ & 16.131  \\
	\addlinespace[0.5em]
	$\widehat{\Sigma}_{22}$ & $\num{8.133e-5}$ & 113.79  & & & $\num{4.156e-5}$ & 9.254    \\
	\addlinespace[0.5em]
	$\widehat{\Sigma}_{33}$ & $\num{2.108e-3}$ & 23.52   & & & $\num{2.941e-3}$ & 6.674   \\
	\addlinespace[0.5em]
	$\widehat{\Sigma}_{12}$ & $\num{-3.662e-8}$ & 251.31 & & & $\num{1.545e-8}$  & 36.164 \\
	\addlinespace[0.5em]
	$\widehat{\Sigma}_{13}$ & $\num{-5.244e-8}$ & 177.81 & & & $\num{2.693e-7}$ & 299.6  \\
	\addlinespace[0.5em]
	$\widehat{\Sigma}_{23}$ & $\num{-1.996e-6}$ & 73.88  & & & $\num{3.435e-5}$ & 549.4  \\\bottomrule
	\addlinespace[0.5em]
	\caption{estimates for the second mixing component and respective estimation errors}
\end{longtable}
\begin{table}
	\centering
	\begin{tabular}{@{}lcccccr@{}} \toprule
		Parameter & MM & $e_{MM}$ & ML & $e_{ML} (\%)$ & EM & $e_{EM}$ (\%)\\ \midrule
		$\widetilde{\lambda}$  & 0.94  & 4.08 & & & 0.9812 & 0.119 \\
		\addlinespace[0.5em]
		$\log L^{\star}$ & $\num{1.3903e+5}$ & & & & $\num{1.438e+5}$ & \\ \bottomrule
		\addlinespace[0.5em]
	\end{tabular}
	\caption{mixing proportion estimate and log-likelihood}
\end{table}
From the tables above we see that the EM method is definitely the most accurate one. Therefore, we decide to adopt it for calibrating the GM model to market data. As far as the MM method is concerned, the formulation given in Section \ref{subsec:MM} relies on the assumption of a common correlation matrix between the two Gaussian regimes. Although this assumption reduces the number of parameter to be estimated, there is empirical evidence (see \cite{Campbell2002}) that this is not the case in global financial markets where correlation between asset classes is actually increased during bear markets. Nonetheless, even if MM is not as accurate as EM, it is still a valuable method since it does require full time series but only their sample statistics. This turns out to be particularly useful when distribution parameters are set via market hypothesis and economic views (e.g. bull market in the next investment period) instead of using historical data.
\section{GH calibration} \label{sec:GH_calibration}
In this section we present a modified EM scheme (the MCECM algorithm) for fitting a GH model to data. In Definition (\ref{def:GH}) we introduced the GH distribution using the so-called $(\lambda,\chi,\psi,\bm{\mu},\bm{\Sigma},\bm{\gamma})$-parametrization. Although this is the most convenient one from a modeling perspective, it comes with an identification issue: the distributions $GH(\lambda,\chi,\psi,\bm{\mu},\bm{\Sigma},\bm{\gamma})$ and $GH(\lambda,\chi /k,k\psi,\bm{\mu},k\bm{\Sigma},k\bm{\gamma})$ are the same (it is easily seen by writing the density (\ref{eq:GHdensity}) in the two cases). To solve this problem, we require the mixing random variable $W$ (see Definition (\ref{def:GH})) to have expectation equal to 1. From Equation (\ref{eq:GIG_moment}) we have 
\[ \mathbb{E}[W]=\sqrt{\dfrac{\chi}{\psi}}\frac{K_{\lambda+1}\big(\sqrt{\chi\psi}\big)}{K_{\lambda}\big(\sqrt{\chi\psi}\big)} = 1  \] 
and if we set $\bar{\alpha} = \sqrt{\chi\psi}$ it follows that 
\begin{equation}\label{eq:PsiChi_function_alpha}
\psi=\bar{\alpha}\frac{K_{\lambda+1}\big(\bar{\alpha}\big)}{K_{\lambda}\big(\bar{\alpha}\big)}, \qquad \chi = \frac{\bar{\alpha}^2}{\psi} = \bar{\alpha}\frac{K_{\lambda}\big(\bar{\alpha}\big)}{K_{\lambda+1}\big(\bar{\alpha}\big)}
\end{equation}
The relations above define the $(\lambda,\bar{\alpha},\bm{\mu},\bm{\Sigma},\bm{\gamma})$-parametrization, which will be used in the MCECM algorithm.

Let $\bm{X} \sim GH_m(\lambda,\chi,\psi,\bm{\mu},\bm{\Sigma},\bm{\gamma})$ and $\{\bm{x}_1,\ldots,\bm{x}_n\}$ be a realization of an iid random sample. Our objective is to find an estimate of the parameters represented by $\bm{\theta}=[\lambda,\chi,\psi,\bm{\mu},\bm{\Sigma},\bm{\gamma}]^T$. The log-likelihood function to be maximized is
\begin{equation}\label{eq:ML_function}
\log L(\bm{\theta};\bm{x}) = \log L(\bm{\theta};\bm{x}_1,\ldots,\bm{x}_n) = \sum_{i=1}^{n}\log f_{\bm{X}}(\bm{x}_i;\theta)
\end{equation}
where $f_{\bm{X}}$ is the function in (\ref{eq:GHdensity}). It well-known that finding a maximizer of (\ref{eq:ML_function}) might be difficult, therefore we resort to a different approach. The situation would look much better if we could observe the latent mixing variables $W_1,\ldots,W_n$. Let us suppose to be in this fortunate situation and define the augmented log-likelihood function
\begin{align}\label{eq:augmentedMLfunction}
\log \widetilde{L}(\bm{\theta};\bm{x}_1,\ldots,\bm{x}_n,W_1,\ldots,W_n)& = \sum_{i=1}^{n}\log f_{\bm{X}\lvert W}(\bm{x}_i\lvert W_i;\bm{\mu},\bm{\Sigma},\bm{\gamma}) + \\\nonumber
& \quad + \sum_{i=1}^{n}\log h_{W}(W_i;\lambda,\chi,\psi)
\end{align}
where we used the fact that $f_{(\bm{X}_i,W_i)}(\bm{x},w;\bm{\theta})= f_{\bm{X}_i\lvert W_i}(\bm{x}\lvert w;\bm{\mu},\bm{\Sigma},\bm{\gamma})h_{W_i} (w;\lambda,\chi,\psi) $ and $h_{W_i}$ is the density in (\ref{eq:GIGdensity}). The advantage of this augmented formulation is that the two terms in (\ref{eq:augmentedMLfunction}) can be maximized separately. Although counter-intuitive, the first term involving the difficult parameters (e.g. a matrix), is the easiest to maximize and it is done analytically; the second term has to be treated numerically instead. To overcome the latency of the mixing variables $W_i$'s, the MCECM algorithm is used. The algorithm consists in alternating an \textit{expectation} step (in which the $W_i$'s are replaced by an estimate deducted from the data and the current parameters estimate) and a \textit{maximization} step (where parameters estimates are updated). Suppose we are at iteration $k$ and $\bm{\theta}^{(k)}$ is the current parameters estimate, the two steps are as follows
\begin{itemize}
	\item \textbf{E-step}: compute the conditional expectation of the augmented log-likelihood function given the data and the current parameters estimate 
	\begin{equation}\label{eq:Q}
	Q(\bm{\theta};\bm{\theta}^{(k)}) = \mathbb{E}[\log \widetilde{L}(\bm{\theta};\bm{x},\bm{W})\lvert \bm{x},\bm{\theta}^{(k)}]
	\end{equation}
	\item \textbf{M-step}: maximize $Q(\bm{\theta};\bm{\theta}^{(k)})$ to get $\bm{\theta}^{(k+1)}$.
\end{itemize}

In practice, the E-step amounts to numerically maximize the second term in (\ref{eq:Q}), which is
\begin{align}\label{eq:ML_function_second_term}
&\mathbb{E}\bigg[\sum_{i=1}^{n}\log h_{W_i}(W_i;\lambda,\chi,\psi)\Big| \bm{x},\bm{\theta}\bigg]  = \sum_{i=1}^{n} -\lambda\log \chi + \lambda\log \sqrt{\chi\psi}+ \\\nonumber
&  -\log 2K_{\lambda}(\sqrt{\chi\psi}) + (\lambda-1)\underbrace{\mathbb{E}\big[\log W_i \lvert \bm{x},\bm{\theta}^{(k)}\big]}_{\xi_i}-\tfrac{1}{2}\chi\underbrace{\mathbb{E}\big[W_i^{-1} \lvert \bm{x},\bm{\theta}^{(k)}\big]}_{\delta_i} + \\\nonumber
& -\tfrac{1}{2}\psi\underbrace{\mathbb{E}\big[W_i \lvert \bm{x},\bm{\theta}^{(k)}\big]}_{\eta_i} = n\big(-\lambda\log \chi + \lambda\log \sqrt{\chi\psi}-\log 2K_{\lambda}(\sqrt{\chi\psi})\big) + \\\nonumber
&+(\lambda-1)\sum_{i=1}^{n}\xi_i-\tfrac{1}{2}\chi\sum_{i=1}^{n}\delta_i-\tfrac{1}{2}\sum_{i=1}^{n}\eta_i.
\end{align}
In order to proceed further, we need to compute the conditional expectations $\xi_i,\delta_i$ and $\eta_i$. Thankfully, the following results holds (see Appendix E.1 in \cite{Brey2013})
\[  W_i\lvert \bm{x}_i \sim \mathcal{N}^-\big(\underbrace{\lambda-\tfrac{1}{2}d}_{\widetilde{\lambda}},\underbrace{\chi + (\bm{x}_i-\bm{\mu})^T \bm{\Sigma}^{-1}(\bm{x}_i-\bm{\mu})}_{\widetilde{\chi}},\underbrace{\psi+\bm{\gamma}^T\bm{\Sigma}^{-1}\bm{\gamma}}_{\widetilde{\psi}}\big). \]
By using Equations (\ref{eq:GIG_moment}) and (\ref{eq:GIG_log}) we end up with
\begin{align}\label{eq:delta}
\delta_i & = \mathbb{E}[W_i^{-1}\lvert \bm{x},\bm{\theta}^{(k)}] = \Big(\frac{\widetilde{\chi}}{\widetilde{\psi}}\Big)^{-\frac{1}{2}}\frac{K_{\lambda-1}(\sqrt{\widetilde{\chi}\widetilde{\psi}})}{K_{\lambda}(\sqrt{\widetilde{\chi}\widetilde{\psi}})} \\[20pt]
\label{eq:eta}
\eta_i & = \mathbb{E}[W_i\lvert \bm{x},\bm{\theta}^{(k)}] = \Big(\frac{\widetilde{\chi}}{\widetilde{\psi}}\Big)^{\frac{1}{2}}\frac{K_{\lambda+1}(\sqrt{\widetilde{\chi}\widetilde{\psi}})}{K_{\lambda}(\sqrt{\widetilde{\chi}\widetilde{\psi}})}\\[20pt]
\label{eq:csi}
\xi_i & = \mathbb{E}[\log W_i\lvert \bm{x},\bm{\theta}^{(k)}]= \frac{\mathrm{d}}{\mathrm{d}\alpha}\Bigg\{\Big(\frac{\widetilde{\chi}}{\widetilde{\psi}}\Big)^{\frac{\alpha}{2}}\frac{K_{\lambda+\alpha}(\sqrt{\widetilde{\chi}\widetilde{\psi}})}{K_{\lambda}(\sqrt{\widetilde{\chi}\widetilde{\psi}})}\Bigg\}_{\alpha=0}
\end{align}
We have now all the ingredients to present the MCECM algorithm as exposed in \cite{Brey2013}
\begin{algorithm}[H]
	\caption{MCECM}
	\begin{algorithmic}[1]
		\State Select reasonable starting points. For instance $\lambda^{(1)}=1,\bar{\alpha}^{(1)}=1$, $\bm{\mu}^{(1)}=$ sample mean, $\bm{\Sigma}^{(1)}=$ sample covariance and $\bm{\gamma}^{(1)} = \bm{0}$
		\State Compute $\chi^{(k)}$ and $\psi^{(k)}$ using (\ref{eq:PsiChi_function_alpha})
		\State Compute the weights $\eta_i$ and $\delta_i$ using (\ref{eq:delta}) and (\ref{eq:eta}). Average the weights to get \[ \bar{\eta}^{(k)}=\frac{1}{n}\sum_{i=1}^{n}\eta_i \quad  \quad \bar{\delta}^{(k)}=\frac{1}{n}\sum_{i=1}^{n}\delta_i \]
		\State If a symmetric model is to be fitted set $\bm{\gamma} = \bm{0}$, else \[ \bm{\gamma}^{(k+1)} = \frac{1}{n}\frac{\sum_{i=1}^{n}(\delta_i^{(k)}(\bar{\bm{x}}-\bm{x}_i)}{\bar{\eta}^{(k)}\bar{\delta}^{(k)}-1} \]
		\State Update $\bm{\mu}^{(k)}$ and $\bm{\Sigma}^{(k)}$ \[ \bm{\mu}^{(k+1)} = \frac{1}{n}\frac{\sum_{i=1}^{n}\delta_i^{(k)}(\bm{x}_i-\bm{\gamma}^{(k+1)})}{\bar{\delta}^{(k)}} \]
		\[ \bm{\Sigma}^{(k+1)}=\frac{1}{n}\sum_{i=1}^{n}\delta_i^{(k)}(\bm{x}_i-\bm{\mu}^{(k+1)})(\bm{x}_i-\bm{\mu}^{(k+1)})^T-\bar{\eta}^{(k)}\bm{\gamma}^{(k+1)} \bm{\gamma}^{(k+1)T}   \]
		\State Set $\bm{\theta}^{(k,2)}=[\lambda^{(k)},\bar{\alpha}^{(k)}\bm{\mu}^{(k+1)},\bm{\Sigma}^{(k+1)},\bm{\gamma}^{(k+1)}]$ and compute $\eta_i^{(k,2)},\delta_i^{(k,2)}$ and $\xi_i^{(k,2)}$ using (\ref{eq:eta}),(\ref{eq:delta}) and (\ref{eq:csi})
		\State Maximize (\ref{eq:ML_function_second_term}) with respect to $\lambda$ and $\bar{\alpha}$ (using relation (\ref{eq:PsiChi_function_alpha})) to complete the calculation of $\bm{\theta}^{(k,2)}$. Go to step 2
	\end{algorithmic}
\end{algorithm}






\chapter{Numerical Results in the Time-Driven Approach}\label{chpt:NumResTD}
This chapter is dedicated to presenting the results obtained by applying the Stochastic Reachability approach (discussed in Chapter \ref{chpt:Model_Description}) to the asset allocation problem. We recall that the output of the \gls{ODAA} algorithm (see Theorem (\ref{thm:rec_algo})) is a sequence of allocation maps $\pi^{\star}=\{\mu_0^{\star},\ldots,\mu_{N-1}^{\star}\}$. For any portfolio realization $x \in \mathbb{R}$ at time $k \in \mathbb{N}$, the maps $\mu_k^{\star}$ provides us with the optimal asset allocation $\mu_k^{\star}(x)=\bm{u}_k^{\star}$; for instance, if $\bm{u}_k^{\star}= \begin{bmatrix}
0.2 & 0.2 & 0.6
\end{bmatrix}^T$, this means that 20\% of investor's wealth should be allocated to the first asset class, 20\% to the second one and the remaining 60\% to the third one. Objective of this chapter is to see what form these maps have at different time instants. The chapter unfolds as follows: in Section \ref{sec:The_Dataset} the dataset is presented and summarized by some sample statistics, in Section \ref{sec:Allocation_Maps} the parameters of the asset allocation problems are set and the allocation maps for the \gls{GM} model are reported. Moreover, the \gls{ODAA} strategy will be compared with other famous asset allocation strategies such as the Constant-Mix and the \gls{CPPI}.

\section{The Dataset}\label{sec:The_Dataset}
Our asset class menu consists of cash, bond and equity. To represent these markets we adopt the indexes presented in Table \ref{tab:indexes}.
\begin{table}[]
	\centering
	\begin{tabular}{@{}lll@{}} \toprule
		Label & Asset Class & Index\\ \midrule
		C & Money Market & iShares Short Treasury Bond ETF\\
		\addlinespace[0.5em]
		B & US Bond  & Northern US Treasury Index \\
		\addlinespace[0.5em]
	    E &	US Equity &  {S\&P 500}\\ \bottomrule
		\addlinespace[0.5em]
	\end{tabular}
	\caption{Asset class and relative index}
	\label{tab:indexes}
\end{table}
The dataset is composed of weekly time series from 23 January 2010 to 15 April 2016. The data is downloaded from Yahoo Finance\footnote{\url{https://finance.yahoo.com/}.}, which is also where the reader is referred for more details of index composition. An overview of the asset classes is given in Figure \ref{fig:assetclassReturns} and Table \ref{tab:sampleStatistics}. By comparing the annualized mean return, it is clear that asset class Equity leads to higher performance than Bond and Bond, in turn, ensures higher performance than Cash. However, the annualized volatility tells us that the same hierarchy holds true also in terms of riskiness, being Equity the riskiest investment and Cash the least risky. Higher sample moments (Skewness and Kurtosis) suggest that the return distribution diverges significantly from a multivariate Gaussian. Indeed, a quantitaive proof of this fact is given us by the Henze-Zirkler\footnote{See \cite{NormTest} for a MATLAB implementation.} multivariate normality test which exhibits a zero p-value. 
\begin{figure}[h]\label{fig:assetclassReturns}
	\centering
	\includegraphics[scale=0.6]{Images/ReturnsHist.png}
	\caption{Weekly asset class returns histogram.}
\end{figure}
\begin{table}[h]
	\centering
	\begin{tabular}{@{}llll@{}} \toprule
		Statistic & C & B & E \\ \midrule
		Mean Return (ann) & 0.064\%  & 3.46\% & 12.11\%\\
		\addlinespace[0.5em]
		Volatility (ann) & 0.113\%  & 4.81\% & 14.81\% \\
		\addlinespace[0.5em]
		Median (ann) &	0\% & 4.58\% & 17.74\% \\
		\addlinespace[0.5em]
		Skewnwss & 0.262 & -0.0621 & -0.36 \\
		\addlinespace[0.5em]
		Kurtosis & 3.90 & 10.62 & 4.42 \\
		\addlinespace[0.5em]
		Monthly $V@R_{0.95}$ & 0.0808\% & 3.73\% & 14.95\%\\
		\addlinespace[0.5em]
		Max Drawdown & 0.106\% & 5.87\% & 23.98\% \\
		\addlinespace[0.5em]
		Mean Drawdown & 0.020\% & 1.5\% & 4.62\% \\
		\addlinespace[0.5em]
		Sharpe ratio & 0 & 0.692 & 0.767 \\ \bottomrule
		\addlinespace[0.5em]
	\end{tabular}
	\caption{Asset class returns sample statistics}
	\label{tab:sampleStatistics}
\end{table}

Finally, the sample correlation matrix is  
\[ 
\begin{bmatrix}
1 & 0.166 & -0.075 \\
  &  1    & -0.454 \\
  &       &  1
\end{bmatrix}.
\]

\section{Optimal Allocation Maps}\label{sec:Allocation_Maps}
Let us consider an investment characterized by the following parameters:
\begin{itemize}
	\item 2-year investment horizon
	\item weekly rebalancing frequency, which means $N=104$ portfolio rebalancings
	\item monthly (\textit{ex-ante}) value-at-risk equals to 7\%
	\item target return $\theta=7\%$ per year
	\item initial wealth $x_0 = 1$.
\end{itemize}
The target sets we want our portfolio value to stay within are 
\begin{align*}
X_0 & = \{1\}\\
X_k & = [0,\infty) \quad k = 1,\ldots,103 \\
X_{104} & = [(1+\theta)^2,\infty) = [1.07^2,\infty).
\end{align*}
In practice, these sets are discretized with a discretization step of $10^{-3}$ and truncated where the probability measure is negligible; the actual sets used in the implementation thus are $X_k = [0.5,1.9]$ $k=1,\ldots,103$ and $X_{104}=[(1.07)^2,1.9]$. As stated in Problem \ref{prb:ODAA}, we are looking for a sequence of allocation maps which maximize the following joint probability
\[ \mathbb{P}\big(\{\omega \in \Omega : x_0 \in X_0,\ldots,x_{104} \in X_N \} \big).\]
The final choice to be made before running the \gls{ODAA} algorithm is picking a model for the asset class returns. As an example, we opt for the \gls{GM} model, which has been fitted to data applying the \gls{EM} method (see Subsection \ref{subsec:EM}). The parameter estimates are:

\begin{align}
\label{eq:GMparam1}
\bm{\mu}_1 & = 
\begin{bmatrix}
\num{1.054e-5} \\
\num{3.713e-4} \\
\num{2.298e-3}
\end{bmatrix}
\quad & \bm{\Sigma}_1 &= 
\begin{bmatrix}
\num{2.437e-8} & \num{1.266e-7} & \num{-2.365e-7} \\
& \num{3.596e-5}  & \num{-5.944e-5} \\
&                & \num{4.232e-4}
\end{bmatrix} \\
\bm{\mu}_2 & = \begin{bmatrix}
\num{2.115e-4} \\
\num{3.105e-2} \\
\num{-8.266e-3}
\end{bmatrix}
\quad & \bm{\Sigma}_2 &= 
\begin{bmatrix}
\num{2.372e-8} & \num{-7.961e-7} & \num{1.277e-6} \\
& \num{2.9e-5}  & \num{-4.411e-5} \\
&                & \num{6.949e-5}
\end{bmatrix}
\label{eq:GMparam2}
\end{align}
and $\lambda = 0.9908$.
By applying the backward algorithm enunciated in Theorem \ref{thm:rec_algo}, we obtained 103 allocation maps; some of which are reported in Figure \ref{fig:mapsMixture}.
\begin{figure}[]
	\makebox[\textwidth][c]{\includegraphics[scale =1]{Images/mapsMixturewk.png}}
	\caption{Optimal allocation maps, weekly rebalancing, GM model}
	\label{fig:mapsMixture}
\end{figure}


Let us now take the time to analyze the kind of investment strategy these maps imply. At the beginning of the investment ($k=0$), the optimal strategy prescribes that 25\% of investor's wealth be invested in Bond and 75\% in Equity. After 25 weeks, depending on the realization of portfolio value (x-axis in Figure \ref{fig:mapsMixture}), the optimal strategy tells us to allocate wealth as follows: if the portfolio is underperforming (e.g. its value is approximately below 1.029), the optimal allocation is a mix of Equity and Bond, which is the riskiest mix allowed (a 100\% allocation in Equity is not permitted due to the \textit{risk} constraint). As soon as performance gets better (i.e. from 1.029 to 1.16) the Equity weight starts decreasing in favor of more Bond and from a certain point on, also Cash. When the portfolio is doing well (i.e. above 1.16), the whole wealth is invested in Cash, namely the least risky of the three asset classes. We could synthesize this behavior by saying that risky positions are taken when the portfolio is doing poorly, whereas conservative positions are taken when the portfolio is doing well. This kind of investing strategy is known in the literature with the name of \textit{contrarian} strategy. The name stems from the fact that contrarian investors bet against the prevailing market trend, namely they try to sell "high" and buy "low". Contrarian strategies perform well in volatile markets and poorly in trending market due to their \textbf{concave} nature (see \cite{Perold1988}).
The optimal strategy obtained by the \gls{ODAA} algorithm exhibits the same pattern also at successive rebalancing times, the only difference is that it becomes more extreme while approaching the investment end; for instance, at time $k=103$ there is no transition from the riskiest allocation to the least risky one. In this case, intermediate position are discarded since either the target has already been reached (hence a 100\% Cash position) or it has not (hence the riskiest position).

 The joint probability of reaching investor's goal is $J(x_0) = p^{\star} = 78.72\%$. This result is verified by running a Monte-Carlo simulation with $10^5$ draws at each rebalancing period from a \gls{GM} distribution with parameters (\ref{eq:GMparam1}) and (\ref{eq:GMparam2}). The joint probability obtained is $p_{MC} = 78.73\%$. Another interesting feature of the \gls{ODAA} strategy is that $p^{\star}$ increases as the rebalancing frequency decreases. By looking at Table \ref{tab:ODAA_results}, it can be seen that as we move from a quarterly rebalancing frequancy\footnote{the problem of switching from a rebalancing frequency to another has been tackled as follows: the model is calibrated to weekly data, then linear returns are approximated by log-returns enabling us to write additive relations such as $w_{monthly} = w_{wk1}+\ldots+w_{wk4}$. Finally, using the hypothesis of iid returns we analytically derive the distribution of monthly and quarterly returns for the G, GM and NIG model. All this models are closed under convolution. } to a monthly one the optimal probability goes from 69.44\% to 73.20\%, and the same happens from monthly to weekly. This fact is rather intuitive since the more rebalancings the more chances to steer the portfolio within the target sets. It should be noted however, that in practice transaction costs have not a negligible impact on portfolio profitability when rebalancing is frequent. This is the reason why investment policies that update portfolio weights only when an "event" occurs are particularly appealing (they will be treated in Part \ref{part:2}).
 
 
\begin{table}[]
	%\renewcommand{\arraystretch}{0.5}
	\centering
	\resizebox{\textwidth}{!}{\begin{tabular}{@{}*{10}{c}@{}}
		\toprule
		& \multicolumn{3}{c}{G} & \multicolumn{3}{c}{GM} & \multicolumn{3}{c}{NIG} \\
		\addlinespace[0.5em]
		\cmidrule(l){2-4} \cmidrule(l){5-7} \cmidrule(l){8-10} 
		& wk & m & q 	& wk & m & q 	& wk & m & q\\
		\addlinespace[0.5em]
	$p^{\star}$ & 79.67\% & 75.56\% & 73.26\% & 78.59\%  & 73.20\% & 69.44\% & 78.53\% & 73.24\% &69.47\%  \\
	\addlinespace[0.5em]
	$p_{MC}$ & 79.77\% & 75.56\% & 73.28\% & 78.82\%  & 73.21\% & 69.58\% & 78.76\% & 73.30\% &69.32\%  \\
	\addlinespace[0.5em]	
	time[h] & 0.712 & 0.157 & 0.050 & 0.857  & 0.316 & 0.283 & 6.131 & 1.467 &0.371  \\	\bottomrule
	\end{tabular}}
	\caption{Probability of reaching the target set obtained via  ODAA algorithm ($p^{\star}$) and Monte-Carlo simulation ($p_{MC}$) for the Gaussian, \gls{GM} and \gls{NIG} model and different rebalancing frequencies (weekly, monthly and quarterly). Time is the computational time of the \gls{ODAA} algorithm, in hours.}
	\label{tab:ODAA_results}
\end{table}

\begin{table}[]
	\centering
	\begin{tabular}{@{}*{4}{c}@{}}
		\toprule
		& G & GM & NIG \\
		%\addlinespace[0.5em]
		\midrule	
		$\log L^{\star}$ & 4396.2& 4471.0  & 4450.2\\
		\addlinespace[0.5em]	
		\bottomrule
	\end{tabular}
	\caption{Log-likelihood for G, GM and NIG model}
	\label{tab:LogL_models}
\end{table}

Next, we used also the Gaussian and the \gls{NIG} model to describe the asset class returns distribution. From Table \ref{tab:LogL_models} we see that the best fitting is provided by the GM model since it exhibits the highest log-likelihood function value; nonetheless, the \gls{NIG} comes right after it. It is not surprising that the Gaussian model is ranked last as we were well-aware that the data considered deviates from a multivariate Gaussian sample (see Table \ref{tab:sampleStatistics}).
%\begin{wraptable}{l}{6cm}
%	\centering
%	\begin{tabular}{@{}*{5}{c}@{}}
%		\toprule
%	Moment	& G & GM & NIG & Empirical\\
%		\addlinespace[0.5em]
%		\midrule	
%	$\mu_{x_{k+1}}$	& & & \\
%	\addlinespace[0.5em]	
%	$\sigma_{x_{k+1}}$	& & & \\
%	\addlinespace[0.5em]
%	$\gamma_{x_{k+1}}$	& & & \\
%	\addlinespace[0.5em]
%	$\kappa_{x_{k+1}}$	& & & \\
%	\addlinespace[0.5em]
%	\bottomrule
%	\end{tabular}
%\end{wraptable}


\subsection{ODAA vs CPPI vs Constant-Mix}
Within the class of asset allocation strategies, the \gls{CPPI} and the Constant-Mix are among the most popular ones (see \cite{Perold1988}). After briefly discussing how they work, we will compare their performance to the ODAA's one.
\paragraph{CPPI}
The idea behind the \gls{CPPI} is to maintain the portfolio \textbf{exposure} to the risky asset, $E_k$, equal to a constant multiple $m$ of the portfolio \textbf{cushion}, $C_k$. The risky asset is assume to be a mix of Bond and Equity. The cushion at time $k$ is defined as
\[
C_k = \max\big\{x_k-F_k,0 \big\}
\]
where $x_k$ is the portfolio value at time $k$ and $F_k$ is the so-called portfolio \textbf{floor}. The floor is a value below which the investor does not want the portfolio value to fall. In our case, the floor is a risk-free asset which grows deterministically at the Cash rate. Therefore, once the investor has specified 
\begin{itemize}
	\item an initial allocation $\bm{u}_0$ 
	\item the initial floor $F_0$ 
	\item a cushion multiplier $m$
	\item the maximum value-at-risk ($V@R_{1-\alpha}$) according to his risk profile,
\end{itemize}
he can synthesized the \gls{CPPI} strategy as follows
\begin{equation*}
\begin{aligned}
& \underset{\bm{u}_k}{\text{maximize}} & &  A\bm{u}_k \\
& \text{subject to} & & \bm{u}_k^T\bm{1}=1, \\
& & & u_i \geq 0 \qquad \forall i \in \{1,2,3\},\\
& & &\bm{u}_k\bm{\Lambda}\bm{u}_k \leq \sigma_{max}^2,\\
& & & \underbrace{xA\bm{u}_k}_{E_k} \leq mC_k.
\end{aligned}
\end{equation*}
where $A=\begin{bmatrix}0 & 1 & 1\end{bmatrix}$, $\sigma_{max}=\frac{V@R_{1-\alpha}}{z_{1-\alpha}}$, $x \in X_k$ and $k = 1,\ldots,N$. From this formulation we see that the investor aims at maximizing the allocation in the risky asset (matrix $A$ selects the allocation in Bond and Equity) while keeping under control the riskiness of the overall allocation and limiting the risky exposure to $m$ times the cushion. The covariance matrix $\bm{\Lambda}$ depends on the model chosen to describe the asset class returns distribution. In our analysis, we set $m=6$, $\bm{u}_0$ equal to the initial ODAA allocation, $F_0$ is chosen in such a way that also the initial exposure is $m$ times the initial cushion and the asset class return random vector follows a GM distribution with parameters (\ref{eq:GMparam1}) and (\ref{eq:GMparam2}). The others investment parameters are equal to the ones in the ODAA case. The CPPI allocation maps are reported in Figure \ref{fig:mapsMixtureCPPI}.



\paragraph{Constant-Mix}
Following a Constant-Mix strategy means maintaining an exposure to the risky asset that is a constant proportion of wealth. For instance, suppose one decides to keep a 60/40 proportion between risky and risk-free asset. After a rebalancing time, asset prices change causing the portfolio proportion to change as well. Let us suppose that the risky asset has increased its price while the risk-free has fall. At the next rebalancing time, the Constant-Mix policy prescribes to sell shares of the risky asset and buy shares of the risk-free in order to recover the 60/40 mix. The constant mix is chosen by solving the following equivalent formulation of the Markowitz problem
\begin{equation*}
\begin{aligned}
& \underset{\bm{u}}{\text{maximize}} & &  \bm{u}^T\bm{\mu} \\
& \text{subject to} & & \bm{u}^T\bm{1}=1, \\
& & & u_i \geq 0 \qquad \forall i \in \{1,2,3\},\\
& & &\bm{u}\bm{\Lambda}\bm{u} \leq \sigma_{max}^2.\\
\end{aligned}
\end{equation*}
where $\sigma_{max}=\frac{V@R_{1-\alpha}}{z_{1-\alpha}}$ and $\bm{\mu}$ and $\bm{\Lambda}$ are the mean and the covariance matrix of the random vector $\bm{w}_{k+1}$ which represents the asset class returns.
The distribution of $\bm{w}_{k+1}$ could be any of the ones discussed in Chapter \ref{chpt:assetclass_returns}. The optimization problem has been solves assuming a GM distribution with parameters (\ref{eq:GMparam1}) and (\ref{eq:GMparam2}); the optimal constant mix is
\[ \bm{u}^{\star} = \begin{bmatrix} 0 & 0.2352 & 0.7648 \end{bmatrix}^T.\]


The ODAA and Constant-Mix belongs to the \textbf{concave} allocations strategies (see \cite{Plasse2013}). This means that their policy is to buy risky assets (Equity and Bond) when they fall and sell them when they raise. Concave strategies perform well in oscillating markets. Conversely, the \gls{CPPI} is an example of a \textbf{convex} strategy: risky assets are bought when they raise and sold when they fall. This behavior is clear from the allocation maps in Figure \ref{fig:mapsMixtureCPPI}. When portfolio performance is up, a more risky position is taken, when it is down, risky assets are sold and a more covered position is taken. Convex strategies perform well in trending markets.

In order to compare the performance of this three different strategies we ran a Monte-Carlo simulation. Starting from an initial wealth $x_0=1$, $2\times 10^5$ portfolio trajectories are drawn assuming a GM distribution with parameters (\ref{eq:GMparam1}) and (\ref{eq:GMparam2}) for vectors $\bm{w}_{k+1}$. Performance and risk figures are reported in table \ref{tab:MC_statistics}. The empirical density function of the 2-year investment return is reported in Figure \ref{fig:epirical_densities}.








\begin{figure}[]
	\makebox[\textwidth][c]{\includegraphics[scale = 0.8]{Images/mapsCPPIMixturewk.png}}
	\caption{Optimal CPPI allocation maps, weekly rebalancing, GM model. An example of a convex strategy.}
	\label{fig:mapsMixtureCPPI}
\end{figure}


\begin{table}[]
	\centering
	\begin{tabular}{@{}lccc@{}} \toprule
		Statistic & ODAA & CPPI & Constant-mix \\ \midrule
		$p_{MC}$ &  78.93\%    &  52.70\%    &  61.41\%    \\
		\addlinespace[0.5em]
		Mean Return (ann) & 5.82\%  & 7.55\% & 10.05\%\\
		\addlinespace[0.5em]
		Volatility (ann) & 5.00\%  & 7.93\% & 13.21\% \\
		\addlinespace[0.5em]
		Median (ann) &	7.20\% & 7.35\% & 9.41\% \\
		\addlinespace[0.5em]
		Skewnwss & 0.700 & -0.010 & 0.0146 \\
		\addlinespace[0.5em]
		Kurtosis & 6.700 & 3.052 & 3.012 \\
		\addlinespace[0.5em]
		Monthly $V@R_{0.95}$ & 5.49\% & 5.51\% & 9.42\%\\
		\addlinespace[0.5em]
		Max Drawdown & 43.39\% & 25.77\% & 45.80\% \\
		\addlinespace[0.5em]
		Mean Drawdown & 1.97\% & 2.12\% & 4.20\% \\
		\addlinespace[0.5em]
		Sharpe ratio & 1.163 & 0.951 & 0.760 \\ \bottomrule
		\addlinespace[0.5em]
	\end{tabular}
	\caption{Investment performance for strategies ODAA, CPPI and Constant-mix obtained via Monte-Carlo simulation ($2\times 10^5$ replications).}
	\label{tab:MC_statistics}
\end{table}



\begin{figure}[]
	\includegraphics[scale = 0.6]{Images/DensitiesMixturewk}
	\caption{Empirical density functions of the 2-year return for the ODAA, CPPI and Constant-mix strategy.}
	\label{fig:epirical_densities}
\end{figure}
\part{Event-Driven approach}
%----------------------------------------------------------------------------------------
%	THESIS CONTENT - APPENDICES
%----------------------------------------------------------------------------------------

\appendix
\chapter{Probability Distributions} \label{app:B}
In this appendix we give further details about the probability distributions used in part I. Main references are \cite{Brey2013}, \cite{paolella2007intermediate} and \cite{McNeil2005}.
\section{Generalized Inverse Gaussian}
\begin{definition}[Bessel function]
	The modified Bessel function of the third kind (simply called \textbf{Bessel function}) is defined as \[ K_{\nu}(x) = \frac{1}{2}\int_{0}^{\infty}t^{\nu-1}\exp\big\{-\frac{1}{2}x(t + t^{-1})\big\}\mathrm{d}t, \quad x > 0.  \]
\end{definition}
\begin{definition}[Generalized Inverse Gaussian]
	the density of a \textbf{Generalized Inverse Gaussian} (GIG) random variable $W$ $(W \sim \mathcal{N}^-\big(\lambda,\chi,\psi \big))$ is 
	\begin{equation}\label{eq:GIGdensity}
	f_{GIG}(w) = \Big(\frac{\psi}{\chi}\Big)^{\frac{\lambda}{2}}\frac{w^{\lambda-1}}{2K_{\lambda}(\sqrt{\chi\psi})}\exp\Big\{-\frac{1}{2}\big(\frac{\chi}{w}+\psi w\big)\big)\Big\} 
	\end{equation}
	with parameters satisfying 
	\[
	\begin{cases}
	\chi >0,\psi \geq 0, \quad \text{if} \quad \lambda <0\\
	\chi >0,\psi > 0, \quad \text{if} \quad \lambda =0\\
	\chi \geq0,\psi > 0, \quad \text{if} \quad \lambda >0\\
	\end{cases}
	\]
\end{definition}
\paragraph{Useful formulas}
The following formulas are used in the text:
\begin{equation}\label{eq:GIG_moment}
\mathbb{E}[W^n] = \Big(\frac{\chi}{\psi}\Big)^{\frac{n}{2}}\frac{K_{\lambda+n}(\sqrt{\chi\psi})}{K_{\lambda}(\sqrt{\chi\psi})}
\end{equation}
\begin{equation}\label{eq:GIG_log}
\mathbb{E}[\log W] = \Big\{\frac{\mathrm{d}\mathbb{E}[X^{\alpha}]}{\mathrm{d}\alpha}\Big\}_{\alpha=0}
\end{equation}
\section{Density Functions}
We give here the probability density function for the general multivariate GH distribution and same special cases
\subsection{GH}
\begin{equation}
f(\bm{x}) = c\frac{K_{\lambda-\frac{m}{2}}\Big(\sqrt{\big(\chi+Q(\bm{x})\big)\big(\psi+\bm{\gamma}^T\bm{\Sigma}^{-1}\bm{\gamma} \big)}\Big)\exp\big\{(\bm{x}-\bm{\mu})^T\Sigma^{-1}\bm{\gamma}\big\}}{\Big(\sqrt{\big(\chi+Q(\bm{x})\big)\big(\psi+\bm{\gamma}^T\bm{\Sigma}^{-1}\bm{\gamma} \big)}\Big)^{\frac{m}{2}-\lambda}}
\end{equation}
where $Q(\bm{x})=(\bm{x}-\bm{\mu})^T\bm{\Sigma}^{-1}(\bm{x}-\bm{\mu})$ and \[ c=\frac{\big(\sqrt{\chi\psi}\big)^{-\lambda}\psi^{\lambda}\big(\psi+\bm{\gamma}^T\bm{\Sigma}^{-1}\bm{\gamma} \big)^{\frac{m}{2}-\lambda}}{(2\pi)^{\frac{m}{2}}\lvert\bm{\Sigma}\lvert^{\frac{1}{2}}K_{\lambda}(\sqrt{\chi\psi})}  \]
\subsection{Student-t}
Setting the degree of freedom $\nu=-2\lambda$ the density reads
\begin{equation}
f(\bm{x}) = c\frac{K_{\frac{\nu+m}{2}}\Big(\sqrt{\big(\nu-2+Q(\bm{x})\big)\big(\bm{\gamma}^T\bm{\Sigma}^{-1}\bm{\gamma}\big)}\Big)\exp\big\{(\bm{x}-\bm{\mu})^T\Sigma^{-1}\bm{\gamma}\big\}}{\Big(\sqrt{\big(\nu-2+Q(\bm{x})\big)\big(\bm{\gamma}^T\bm{\Sigma}^{-1}\bm{\gamma}\big)}\Big)^{\frac{\nu+m}{2}}}
\end{equation}
where \[ c = \frac{\big(\nu-2\big)^{\frac{\nu}{2}}\big(\bm{\gamma}^T\bm{\Sigma}^{-1}\bm{\gamma}\big)^{\frac{\nu+m}{2}}}{(2\pi)^{\frac{m}{2}}\lvert\bm{\Sigma}\lvert^{\frac{1}{2}}\Gamma(\frac{\nu}{2})2^{\frac{\nu}{2}-1}} \]
\subsection{VG}
\begin{equation}
f(\bm{x}) = c\frac{K_{\lambda-\frac{m}{2}}\Big(\sqrt{Q(\bm{x})\big(2\lambda+\bm{\gamma}^T\bm{\Sigma}^{-1}\bm{\gamma} \big)}\Big)\exp\big\{(\bm{x}-\bm{\mu})^T\Sigma^{-1}\bm{\gamma}\big\}}{\Big(\sqrt{Q(\bm{x})\big(2\lambda+\bm{\gamma}^T\bm{\Sigma}^{-1}\bm{\gamma} \big)}\Big)^{\frac{m}{2}-\lambda}}
\end{equation}
where \[ c=\frac{2\lambda^{\lambda}\big(2\lambda+\bm{\gamma}^T\bm{\Sigma}^{-1}\bm{\gamma}\big)^{\frac{m}{2}-\lambda}}{(2\pi)^{\frac{m}{2}}\lvert\bm{\Sigma}\lvert^{\frac{1}{2}}\Gamma(\lambda)} \]







\backmatter
%----------------------------------------------------------------------------------------
%	BIBLIOGRAPHY
%----------------------------------------------------------------------------------------

\backmatter
\nocite{*}
\bibliographystyle{acm}
\bibliography{Bibliography/biblio}

\end{document}          
