\section{Interest rate dynamics for the risk-free asset}\label{sec:Interest_rate_dynamics}
In this section we try to extend the model presented in Chapter \ref{chpt:ED} by assuming a dynamics for the interest rate of the bank account. Starting from the portfolio dynamics (\ref{eq:ptf_dynamic_ED}), our aim is replacing it with
\begin{equation}\label{eq:portfolio_dynamic_Vasicek}
\boxed{x_{k+1} = x_k\big( \exp\Big\{\int_{t_k}^{t_k+\tau_{k+1}}\!\!\!\!\!\!\!\!r_s\mathrm{d}s \Big\} + u_kJ\widetilde{N}_{k+1}\big)} \qquad k \in \mathbb{N}
\end{equation}
where $\{r_t\}_{t\geq t_k}$ is the short-rate process and everything else remains unchanged from Chapter \ref{chpt:ED}. Due to its analytical tractability, we decided to model the short-rate according to the Vasicek Model (see \cite{brigo2007}). The following \gls{SDE} provides the dynamics of the short-rate
\begin{equation}\label{eq:VasicekSDE}
\begin{cases*}
dr_t = a(b-r_t)dt + \sigma dW_t\\
r_{t_k} = r_k, \qquad t \geq t_k
\end{cases*}
\end{equation}
where $a,b,\sigma$ and $r_k$ are positive constant and $\{W_t\}_{t\geq t_k}$ an unidimensional Brownian motion. The main feature of the Vasicek model is the \textit{mean reversion} property: the process will tend to move to its average over time. Moreover, the process has a non-null probability to became negative. This is no longer a taboo since negative interest rates are seen in the market.

The solution of \gls{SDE} (\ref{eq:VasicekSDE}) is the following Ornstein–Uhlenbeck process
\begin{equation}\label{eq:solution_VasicekSDE}
r_t = r_ke^{-a(t-t_k)}+b(1-e^{-a(t-t_k)}) + \sigma e^{-a(t-t_k)}\int_{t_k}^{t}e^{a(s-t_k)}dW_s.
\end{equation}
However, we are interested in an explicit expression of the integrated version of process $r_t$ since it appears in the portfolio dynamics. For this reason, let us define the integrated short-rate process by $v_t=\int_{t_k}^{t}r_s\mathrm{d}s$. In order to find its explicit form, we integrate (\ref{eq:VasicekSDE}) from $t_k$ to a generic instant $t$, obtaining
\begin{equation}\label{eq:integrated_VasicekSDE}
r_t = r_k + a\big(b(t-t_k) - v_t\big) + \sigma(W_t-W_{t_k}).
\end{equation}
After equating (\ref{eq:integrated_VasicekSDE}) and (\ref{eq:solution_VasicekSDE}) and solving for $v_t$ we get
\begin{equation}\label{eq:v_t}
v_t = \frac{1}{a}\Big[(r_k-b)(1-e^{-a(t-t_k)})+ab(t-t_k)+\sigma\int_{t_k}^{t}(1-e^{-a(t-s)})dW_s\Big].
\end{equation}
From Appendix \ref{app:IntegratedOU} and (\cite{baldi2017}, Proposition 7.1) we have that
\begin{align}
\nonumber
v_t & \sim \mathcal{N}\Big(\eta(t-t_k),\zeta^2(t-t_k)\Big)\\[2ex]
\label{eq:Vasicek_mean}
\eta(t-t_k) & = \mathbb{E}[v_t]= \frac{1}{a}\Big[(r_k - b)\big(1-e^{-a(t-t_k)}\big)+ab(t-t_k)\Big]\\[2ex]
\label{eq:Vasicek_variance}
\zeta^2(t-t_k) & = \Var{v_t}= \frac{\sigma^2}{2a^3}\Big[2a(t-t_k) + 4e^{-a(t-t_k)}-e^{-2a(t-t_k)}-3\Big]
\end{align}
where $t \in \mathbb{R}^{+}$.
\begin{remark}
	The parameters $\eta(t-t_k)$ and $\zeta(t-t_k)$ depends only on the difference $t-t_k$, therefore the distribution of $v_t$ is stationary. This fact will be particularly useful in the following since the distribution of $v_{t_k+t}$ (which is the one we are interested in) depends only on $t$.
\end{remark}
\subsection{The density of $x_{k+1}$}
The last step is find the probability density function of the random variable $x_{k+1}$. 
\begin{proposition}
	Let $x_{k+1}$ be the random variable (\ref{eq:portfolio_dynamic_Vasicek}). Then, its probability density function is
	\begin{align*}
	f_{x_{k+1}}(z) &=\frac{p}{(z-\xi)}\bigg\{ \int_{0}^{\infty}\varphi\Big(\frac{\log\big(\frac{z-\xi}{x}\big)-\eta(t)}{\zeta(t)}\Big)\Big(\frac{1}{\zeta(t)}\Big)f_{\tau_{k+1}}(t)\mathrm{d}t\bigg\}\mathbbm{1}_{[x+\xi,\infty)} \nonumber +\\
	& + \frac{(1-p)}{(z+\xi)}\bigg\{ \int_{0}^{\infty}\varphi\Big(\frac{\log\big(\frac{z+\xi}{x}\big)-\eta(t)}{\zeta(t)}\Big)\Big(\frac{1}{\zeta(t)}\Big)f_{\tau_{k+1}}(t)\mathrm{d}t\bigg\}\mathbbm{1}_{[x-\xi,\infty)}
	\end{align*}
	where $\xi=xJu_k$, $f_{\tau_{k+1}}= \big\{\lambda e^{-\lambda t}\big\}\mathbbm{1}_{[0,\infty)}$ is the density of random variable $\tau_{k+1}$, $\varphi$ is the density of a standard normal and $\eta(t),\zeta(t)$ are functions (\ref{eq:Vasicek_mean}) and the square root of (\ref{eq:Vasicek_variance}) respectively computed in $t+t_k$.
\end{proposition}
\begin{proof}
	Let us rewrite the portfolio dynamics in the following way
	\[
	x_{k+1}=x\exp\{v_{t_k+\tau_{k+1}}\} + xJu_k\widetilde{N}_{k+1}=Y+\xi\widetilde{N}_{k+1}.
	\]
	The first step of the proof is finding the cdf of Y:
	\begin{align*}
	F_Y(y) & = \mathbb{P}\big(x\exp\{v_{t_k+\tau_{k+1}}\}\leq y\big)\\
	& = \int_{0}^{\infty}\mathbb{P}\Big(v_{t_k+\tau_{k+1}}\leq \log(y/x)\lvert \tau_{k+1}=t\Big)f_{\tau_{k+1}}(t)\mathrm{d}t\\
	& = \int_{0}^{\infty}\mathbb{P}\Big(v_{t_k+t}\leq \log(y/x)\Big)f_{\tau_{k+1}}(t)\mathrm{d}t\\
	& = \Bigg\{\int_{0}^{\infty}\varphi\Big(\frac{\log(y/x)-\eta(t)}{\zeta(t)}\Big)f_{\tau_{k+1}}(t)\mathrm{d}t \Bigg\}\mathbbm{1}_{[x,\infty)}
	\end{align*}
	where we used the Law of Total Probability in the continuous case and the fact that $v_{t_k+t}$ is Gaussian with parameters $\eta(t)$ and $\zeta(t)$ (which are the functions (\ref{eq:Vasicek_mean}) and the square root of (\ref{eq:Vasicek_variance}) respectively computed in $t+t_k$).
	
	
Repeating the usual steps, we are able to find the cdf of $x_{k+1}$:
\begin{align*}
F_{x_{k+1}}(z) & = \mathbb{P}\big(Y+\xi\widetilde{N}_{k+1}\big) \\
& = F_Y(z-\xi)p+F_Y(z+\xi)(1-p)\\
& = p\bigg\{ \int_{0}^{\infty}\Phi\Big(\frac{\log\big(\frac{z-\xi}{x}\big)-\eta(t)}{\zeta(t)}\Big)f_{\tau_{k+1}}(t)\mathrm{d}t\bigg\}\mathbbm{1}_{[x+\xi,\infty)}+\\
& + (1-p)\bigg\{ \int_{0}^{\infty}\Phi\Big(\frac{\log\big(\frac{z+\xi}{x}\big)-\eta(t)}{\zeta(t)}\Big)f_{\tau_{k+1}}(t)\mathrm{d}t\bigg\}\mathbbm{1}_{[x-\xi,\infty)}.
\end{align*}
where $\Phi$ is the cdf of a standard normal and $p$ the probability that $\widetilde{N}_{k+1}$ is equal to 1. Finally, deriving with respect to $z$ under the integral sign gives us the result.
\end{proof}
\subsection{The calibration of the Vasiceck model}
\subsection{Numerical results}



