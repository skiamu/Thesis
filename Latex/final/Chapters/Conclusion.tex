\chapter{Conclusions}\label{chpt:Conclusions}
In this final chapter we summarize what has been achieved in this thesis and then suggest some directions for future research.

\section{The thesis in a nutshell}
Recent results in Stochastic Reachability have been the basis for this thesis. These results (e.g. the \gls{ODAA} theorem) allowed us to cast the asset allocation problem in a Control System setting. The generality of this approach permitted portfolio rebalancings to be driven first by time and then by the occurrence of discrete events. In the time-driven case, a universe of three asset classes (Cash, Bond and Equity) was considered and the allocation maps for a 2-year investment were obtained. These maps exhibited a \textit{contrarian} attitude: the higher the portfolio performance the less risky the asset class mix, the lower the performance the riskier the mix. In the 2-year investment example, the \gls{ODAA} strategy outperformed both the \gls{CPPI} and the constant-mix in terms of annualized returns while showing a comparable risk profile. In the event-driven case, the portfolio consisted of a risk-free asset (a bank account) and a risky asset (an exposure to a future index). In this context, the risky asset allocation maps, obtained via \gls{ODAA} algorithm, showed a \textit{contrarian} behavior in the sense that when portfolio performance is up the risky asset is shorted, when it is down a long position is taken instead.

As far as the time-driven model is concerned, a small contribution to the literature has been the use of Generalized Hyperbolic distributions, as an alternative to the Gaussian Mixture model proposed in \cite{Pola12}, to describe the statistical properties of asset class returns.
The personal contribution to the event-driven literature has been the extension of the model introduced in \cite{specchio2011} in two innovative ways. In particular, first the risky asset has been modeled as a Geometric Brownian Motion, then an interest rate dynamics as been considered for the bank account. In the first case, the usual \textit{contrarian} allocation maps for the risky asset were obtained. In the second case, the complexity of the formulas involved made the implementation quite difficult; a future research direction could be finding suitable numerical methods to tackle these issues. 
\section{Further developments}
The first idea could be casting the asset allocation problem in a stochastic hybrid system setting. A stochastic hybrid system is a control system whose evolution has a continuous and a discrete component. For instance, the problem of maintaining the temperature of $r \in \mathbb{N}$ different rooms within a certain range over time could be modeled as a stochastic hybrid system. The discrete state space being the mode (ON or OFF) of the heater switches and the continuous state space the room temperatures. In an asset allocation problem, the discrete state space could be the states of the economy. Main references are \cite{ABATE2008} for the discrete-time case and \cite{Boj2012} for the continuous-time case.





