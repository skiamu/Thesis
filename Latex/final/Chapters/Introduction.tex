\chapter{Introduction}

Since the first stock exchange opened for trades, investors have been trying to find ways to allocate their wealth among different securities in order to maximize returns. As a matter of fact, the techniques which are being employed nowadays in the investment industry are quite different from those used in the first half of the 20th century. "Fifty years ago, the business of managing other people's money was very much an art not a science, and was largely a matter of finding someone who was privy to inside information. But during the 1950s, 1960s and 1970s, academics changed the study of what became known as portfolio management. They did so in the face of much initial resistance and scepticism from the industry"\footnote{See \cite{Econ}.}. 

\section{The academia meets the industry}
In those days, any market participant would have been aware that investing was a risky business (nothing ventured, nothing gained). However, a formal and systematic connection between risk and return was still missing. The "annus mirabilis" in asset allocation was 1952, when Harry Markowitz published his pioneering paper \textit{Portfolio Selection} in the \textit{Journal of Finance}, starting the academic invasion of the financial industry. In his paper, which is considered to be the starting point of \gls{MPT}, Markowitz outlined for the first time how investors should allocate assets so as to achieve the highest returns given a certain level of risk. After having estimated expected returns and the covariance between each security, an investor, according to Markowitz, has to solve a quadratic programming problem for obtaining the so-called \textit{efficient portfolio frontier}. Any portfolio (a mix of securities) belonging to the frontier is efficient in the sense that it provides the highest expected return for a given level of risk (standard deviation). Moreover, each of these portfolios has the feature of being \text{deversified}. In Markowitz's words\footnote{See \cite{bernstein2012}.}:
\begin{quote}
	A portfolio with sixty different railway securities, for example, would not be as well diversified as the same size portfolio with some railroad, some public utility, mining, various sort of manufacturing, etc. The reason is that it is generally more likely for firms within the same industry to do poorly at the same time than for firms in dissimilar industries. 
\end{quote}

Building on Markowitz's work, the second major breakthrough is the \gls{CAPM} and it was made by another University professor, William Sharpe. The \gls{CAPM}, which appeared in the \textit{Journal of Finance} in 1964, allows one to compute the expected return from an asset in terms of its risk. The risk is divided into two components, namely a \textit{systematic risk} (which is related to the return of the whole market and cannot be eliminated) and a \textit{non-systematic risk} (which is unique to the asset and could be eliminated by diversifying the portfolio) \cite{hull2014}. In spite of the unrealistic assumptions which the \gls{CAPM} is based on \cite{hull2015risk},  it has proved to be a useful tool for portfolio managers. Other key contributions academia made to the industry are the \gls{EMH} (Fama, 1970) and the \gls{BS} model. \gls{EMH} is a theory that states that security prices perfectly reflect all available information in the market and, consequently, the whole market cannot be beaten. On the other hand, the \gls{BS} model is a mathematically-rigorous theory for pricing options.

Although all of the models above have shown some shortcomings when applied to the complex reality of capital markets, they have been crucial steps for reaching what Asset Allocation is today. Nonetheless, ideas from other scientific fields have continued to  enrich Finance and in particular Asset Allocation. An example of this flourishing contamination is the techniques of Stochastic Reachability presented in this works which have been borrowed from the theory of Control Systems and applied to the asset allocation problem.



\section{Structure}
In this section the structure of the thesis is outlined.

In Chapter \ref{chpt:Model_Description} we introduce the asset allocation problem and all the quantities related to it, such us the asset class return vector and the portfolio dynamics. Then, the concept of Stochastic Reachability is explored giving an idea about its possible fields of application. We conclude the chapter by discussing the mathematical formulation of the asset allocation problem in a Stochastic Reachability framework. In particular, the \gls{ODAA} algorithm, which all the thesis rely on, is enunciated.

In Chapter \ref{chpt:assetclass_returns} we discuss three models which could be employed to describe the probabilistic properties of the asset class returns vector, namely the \gls{G}, \gls{GM} and \gls{GH} model. For each of them, the model-related features (\textit{risk} constraint and portfolio value density function) of the \gls{ODAA} algorithm are obtained.

In Chapter \ref{chpt:calibration} we focus on techniques for calibrating the models presented in Chapter \ref{chpt:assetclass_returns} to market data. For the \gls{GM} case, calibration performance between three calibration methods are compared.


Chapter \ref{chpt:NumResTD} is dedicated to presenting the numerical results for the time-driven approach. The \gls{ODAA} asset allocation strategy is compared to the Constant-Mix and the CPPI, which are two benchmark policies in the industry. This will end Part \ref{part:1}

Part \ref{part:2} begins with Chapter \ref{chpt:ED}, where a brief introduction of the \gls{DES} theory is given. Afterwards, the asset allocation problem is cast in an \gls{ED} setting, calibration of model parameters is discussed and numerical results are given. 

Chapter \ref{chpt:Model_Extensions} constitutes the original part of the thesis. In this chapter we attempt to generalize the basic \gls{ED} model of Chapter \ref{chpt:ED} in two different ways: first, by modeling the risky asset as a \gls{GBM} and then by assuming a stochastic dynamics (Vasicek model) for the risk-free interest rate. 


Finally, in Chapter \ref{chpt:Conclusions} we sum up what has been achieved in this thesis and propose future research directions.

The MATLAB code used to implement the models presented in the thesis can be found in the following GitHub repository: \url{https://github.com/skiamu/Thesis}. 











