
\begin{abstract}
	Before the 1950s, managing other people's money was a discipline as far away from being scientifically dictated as it could ever get. For instance, before Markowitz's revolutionary paper \textit{Portfolio Selection}, diversification was not a universally recognized practice in Asset Allocation. Markowitz's work paved the way for new ideas from different scientific and academical fields to influence Finance and Asset Allocation in particular.
	
	Continuing along this line, in this thesis cutting-edge results in Stochastic Reachability (which is a concept belonging to the theory of Control Systems) are employed to tackle the Asset Allocation problem. In particular, once an investor has specified his risk profile and a target return, the model will output an optimal investment strategy having the feature of maximizing the probability of reaching the target return while keeping the risk under control. This strategy will exhibit a \textit{contrarian} behavior, namely it prescribes to buy risky assets (in order to achieve a riskier position) when performance is down and to sell them when performance is up.
	
	
	What are the drivers that lead a portfolio manager to rebalance portfolio weights? In Part I, the case where time triggers a portfolio rebalancing will be explored. Although this Time-Driven approach is the most intuitive, it might incur in non-negligible transaction costs if the rebalancing frequency is high. On the other hand, in Part II, what causes the portfolio mix to be readjusted will be the fact that the risky asset cumulative return hits a lower or upper barrier. This portfolio rebalancing mechanics leads to the so-called Event-Driven approach to Asset Allocation.
	
	\vspace{0.5cm}
	\noindent
	\textbf{Keywords}: Asset Allocation, Stochastic Reachability, Time-Driven approach, Event-Driven approach.

\end{abstract}



\begin{otherlanguage}{italian}
	\begin{abstract}
		Contrariamente a quanto accade oggi per i maggiori investitori istituzionali, prima degli anni '50 chi investiva in borsa lo faceva senza adottare un metodo rigorosamente scientifico. Il cambio di rotta avvenne nel 1952, quando Harry Markowitz, fondatore della moderna teoria del portafoglio, pubblicò \textit{Portfolio Selection}, un articolo che aprì la strada ad un flusso di nuove idee provenienti dall'accademia e da svariate discipline scientifiche, idee che saranno destinate a rivoluzionare il modo di investire nei decenni successivi. 
		
		Proseguendo in questa direzione, nel nostro lavoro vengono utilizzati recenti risultati in Stochastic Reachability (concetto sviluppato nella teoria dei Sistemi di Controllo) per riadattarli in un contesto di Asset Allocation. In particolare, una volta individuato un adeguato profilo di rischio e un rendimento da raggiungere, il modello produrrà una strategia di investimento che massimizza la probabilità di raggiungere tale rendimento tenendo allo stesso tempo sotto controllo il rischio. Questa strategia avrà la caratteristica di essere \textit{Contrarian}, cioè prescrive di acquistare titoli rischiosi quando la performance del portafoglio è buona mentre prescriverà di venderli, in favore di titoli privi di rischio, se la performance è bassa.
		
		Con che criterio dunque, un gestore decide di ribilanciare i pesi di portafoglio? Nella prima parte del lavoro verrà presentato l'approccio Time-Driven, in cui è il tempo a dettare la riallocazione dei pesi (e.g. settimanalmente). Nonostante questo sia il metodo più intuitivo, ha come svantaggio gli elevati costi di transazione nel caso di frequenza di riallocazione alta. Nella seconda parte invece, l'approccio che si segue è quello Event-Driven. Ciò significa che una riallocazione di portafoglio viene effettuata solo quando il valore assoluto del rendimento cumulato dell'asset rischioso supera una certa soglia.
		
		
		\vspace{0.5cm}
		\noindent
		\textbf{Parole chiave}: Asset Allocation, Stochastic Reachability, Time-Driven approach, Event-Driven approach.
		
		
	\end{abstract}
\end{otherlanguage}
